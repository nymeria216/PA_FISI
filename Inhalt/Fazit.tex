% !TEX root = ../Projektdokumentation.tex
\section{Fazit} 
\label{sec:Fazit}

\subsection{Soll-/Ist-Vergleich}
\label{sec:SollIstVergleich}
Das Projektziel wurde erfolgreich innerhalb des vorgegebenen Zeitrahmens erreicht. Während der Planung stellte sich jedoch heraus, dass die 
geschätzte Dauer der Testphase von ursprünglich 6 Stunden zu grob war und stattdessen in 2 Arbeitsstunden abgeschlossen wurde. Daher wurden die 
übrigen 4 Stunden zur Verbesserung der Dokumentation verwendet. Diese Veränderung von 6 auf 2 Stunden in der Projektumsetzung und Dokumentation 
erfolgte. Die geplanten Arbeitsstunden für die beiden beteiligten Teammitglieder wurden voll ausgeschöpft, und die Ressourcen, wie in der 
\nameref{sec:Sachmittelplanung} beschrieben, wurden effizient genutzt.
\\Wie in Tabelle~\ref{tab:Vergleich} \nameref{tab:Vergleich} zu erkennen ist, konnte die Zeitplanung bis auf wenige Ausnahmen eingehalten werden.
\tabelle{Soll-/Ist-Vergleich}{tab:Vergleich}{Zeitnachher.tex}

\subsection{Lessons Learned}
\label{sec:LessonsLearned}
Die Zeitplaung und Schätzung der Testphase erwies sich als zu grob und ungenau. Es ist wichtig zu beachten, realistische Zeitpläne zu erstellen 
und eventuell genügend Puffer einzukalkulieren. Bedingt dessen, dass die Dokumentation ein integraler Bestandteil der Projektarbeit ist, 
wurde die verkürzte Zeit für die testphase sinnvoll für die Verbesserung der Inhalte in der Dokumetation genutzt. Ein Vorteil der 
ausgewählten MFA-Lösung ist, dass die Implementierung recht schnell und nachvollziehbar ging.

\subsection{Ausblick}
\label{sec:Ausblick}
Das Projekt ist eine Tochterprojekt eines übergeordneten Projektes. Mit dem Ergebnis wird die Sicherheit der vorhandenen Dienste gewährleistet. 
Authentik als Identitäts-provider wird weiter ausgebaut, sodass ein Unternehmensdesign zukünftig erstellt wird und Monitoring-Alerts eingefügt, 
die bei fehlerhaften Logins oder -Logouts eine Mail senden.
