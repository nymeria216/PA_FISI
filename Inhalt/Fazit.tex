% !TEX root = ../Projektdokumentation.tex
\section{Fazit} 
\label{sec:Fazit}

\subsection{Soll-/Ist-Vergleich}
\label{sec:SollIstVergleich}
Das Projektziel wurde erfolgreich innerhalb des vorgegebenen Zeitrahmens erreicht. Während der Planung stellte sich jedoch heraus, dass die 
geschätzte Dauer der Testphase von ursprünglich 4 Stunden zu grob war und stattdessen in 3 Arbeitsstunden abgeschlossen wurde. Daher wurde die 
übrige 1 Stunde zur Verbesserung der Dokumentation verwendet. Die geplanten Arbeitsstunden für die beiden beteiligten Teammitglieder 
wurden voll ausgeschöpft, und die Ressourcen, wie in der \nameref{sec:Sachmittelplanung} beschrieben, effizient genutzt.
\\Wie in Tabelle~\ref{tab:Vergleich} \nameref{tab:Vergleich} zu erkennen ist, konnte die Zeitplanung bis auf wenige Ausnahmen eingehalten werden.
\tabelle{Soll-/Ist-Vergleich}{tab:Vergleich}{Zeitnachher.tex}

\subsection{Gewonnene Erkenntnisse}
\label{sec:Gewonnene Erkenntnisse}
Die Zeitplaung und Schätzung der Testphase erwies sich als grob und ungenau. Es ist wichtig zu beachten, realistische Zeitpläne zu erstellen 
und eventuell genügend Puffer einzukalkulieren. Bedingt dessen, dass die Dokumentation ein integraler Bestandteil der Projektarbeit ist, 
wurde die verkürzte Zeit für die Testphase sinnvoll für die Verbesserung der Inhalte in der Dokumetation genutzt. Ein Vorteil der 
ausgewählten \acs{MFA}-Lösung ist, dass die Implementierung recht schnell und nachvollziehbar ging.

\subsection{Ausblick}
\label{sec:Ausblick}
Das Projekt ist eine Tochterprojekt eines übergeordneten Projektes. Mit dem Ergebnis wird die Sicherheit der vorhandenen Dienste gewährleistet 
\bzw verbessert. Authentik als Identitäts-Provider wird erweitert, um die Entwicklung eines Unternehmensdesigns zu ermöglichen. 
Außerdem werden Monitoring-Alerts integriert, um bei fehlerhaften Logins oder Logouts automatisch Benachrichtigungen per E-Mail zu versenden.