% !TEX root = ../Projektdokumentation.tex
\section{Fazit} 
\label{sec: Fazit}

\subsection{Vergleich beider Reden}
\label{sec: Vergleich beider Reden}

Wie in der \nameref{sec: Antrittsrede von Donald Trump} und \nameref{sec: Coronarede von Angela Merkel} bereits dargelegt, gibt es verschiedene rhetorische Mittel, um einen Zweck zu verfolgen und durch die Art des Gebrauches eine bestimmte Zuhörerschaft anzusprechen. \\Beide Personen versuchen mit Ihrer Rede ein Gemeinschaftsgefühl zu erzeugen, was auf unterschiedlichen Art und Weisen passierte. Das Interessante ist weiterhin, dass sich Angela Merkel an die gesamte Bevölkerung der Deutschen Bundesrepublik durch alle gesellschaftlichen Schichten wendet, während Donald Trump scheinheilich die Bevölkerung aber vornehmlich dessen Wählerschaft anspricht.
\\Donald Trump versucht durch die häufige Verwendung des Personalpronomens „wir“ und das Aufzeigen von bestehenden Problemen ein Gefühl der Zusammengehörigkeit zu erzeugen, wobei er die zukünftige Lösung der Vereinigten Staaten darstellen soll. Durch das Verwenden von Anaphern und überzogenen Metaphern nutzt er die Ängste der breiten Masse aus, und verschärft Argumentation mit überspitzenden Hyperbeln. Was seiner Wählerschaft zugutekommt, ist die Verwendung der parataktischen Redeart und häufige Parallelismen,  welche für eine leichte Verständlichkeit sorgen. Durch indirekte Beleidigungen und das Provozieren der Masse, scheint es hier um einen Wettkampf und keine motivierende Antrittsrede zu gehen. Die vereinfachte Darstellung in Wahrheit und Unwahrheit bzw. „schwarz und weiß“ und das Verpacken in Superlative, sind resultierend daraus unsachliche Züge zu erkennen. Seinen auf den ersten Blick einfacher Sprachgebrauch, wählt er ganz gezielt, um die Bevölkerung zu begeistern. Denn eines darf in dem ganzen Kontext nicht vergessen werden, Donald Trump stammt selbst aus der Oberschicht, der politischen Elite und geht es in diesem Punkt ausschließlich um das Machtverhältnis. \\Die klassische Antrittsrede, welche nach Basil Bernstein, eher dem elaborierten Code zuzuordnen ist, hat hier keine Anwendung gefunden. Er nutzt die Gesamtbreite der rhetorischen Mittel nicht um mit Überzeugungskraft zu punkten, sondern um sie manipulativ einzusetzen.

\subsection{Schlussfolgerung}
\label{sec: Schlussfolgerung}
Um ein erfolgreiches Fazit ziehen zu können, begleitet uns zu Beginn der Einleitung die bereits erwähnte 
Leitfrage \textit{„In wie weit beeinflussen rhetorische mittel in politischen Reden das Verhalten des 
Publikums?“}. Diese kann größtenteils mit \textit{"JA"} beantwortet werden. Es ist möglich, durch die Verwendung von 
rhetorischen Mitteln, Auswirkungen auf das Verhalten des Publikums zu haben. Dieses kann eine Emotionale 
Bindung aufbauen, das Verständnis und die Beweggründe fördern, das Publikum dazu bewegen bestimmte 
Handlungen nachzuvollziehen und zu ergreifen. \\Auf der anderen Seite können sie auch genutzt werden, 
um die Zuhörer zum Nachdenken anzuregen und vielleicht einige Handlungen und Entscheidungen kritisch 
zu hinterfragen. Natürlich hängt dies von den Fähigkeiten des Redners selbst ab und welche 
Strategie dieser befolgt, denn auch unterschiedliche rhetorische Mittel haben eine unterschiedliche 
Wirkung auf das Publikum.