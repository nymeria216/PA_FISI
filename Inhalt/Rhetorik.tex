% !TEX root = ../Projektdokumentation.tex
\section{Rhetorik} 
\label{sec:Rhetorik}

\subsection{Was ist Rhetorik?}
\label{sec: Was ist Rhetorik?}
Der Begriff Rhetorik unterteilt sich in eine ältere und modernere Definition.
Zum Einen ist die Rhetorik ein in der Kunst gesteigerte Umgang mit dem Wort selbst.
Er stammt aus dem Altgriechischen \textit{"rhetorike"} und bedeutet in das Deutsche 
übersetzt \textit{"Redekunst"} oder \textit{"Kunst der Beredsamkeit"}.
So stellt sich die Frage: \textit{"Bleibt diese Bedeutung auch in unserer Zeit relevant?"}
Ein Antwort wäre, dass im Laufe der Jahre die Rhetorik eine 
neue Bedeutung erworben hat. Sie bewirkt unter anderem eine Zunahme der sprachlichen 
Flexibilität und unterstützt bei Verhandlungen und Diskussionen. Dadurch ist 
zusammenfassend zu sagen, dass das der aktive zielbewusste Umgang mit Worten ist. Mit der 
Verwendung von Rhetorik, kann das persönliche, als auch das gesamte Umfelt überzeugt werden, 
wenn man selbst ein guter 
Rhetoriker ist. Aus diesem Grund versuchen Politiker sich auf diesem Gebiet einige 
Fähigkeiten anzueignen. So wollen die Zuhörer den Reden dieser gerne folgen und es 
bleibt ihnen lange im Gedächtnis. Sobald es Rhetorik in Präsentationen Verwendung findet, 
erfolgt ein monologischer Übergang, doch kann es im Gespräch oder bei Verhandlungen ebenfalls 
Gebrauch finden, sodass diese Art dialogisch einkategorisiert wird. 
\\Rhetorik ist eine Kunstform und dadurch eine angewandte Wissenschaft. Die Art und Weise eine wirkungsvolle 
Rede zu halten, bildet den wissenschaftlichen Teil, der sich hauptsächlich mit 
der Methode und den Stilmitteln der Rede beschäftigt. Das Verfassen einer Rede ist nicht nur 
mental anstrengend, sondern auch ein kreativer Akt. Dies gilt unter anderem für Reden 
vor einem Publikum, bei denen eine rhetorische Präsenz erforderlich ist und Charisma und Flair eine große Rolle 
für die Wirkung spielen. 
\\Die Kunst der Rhetorik besteht schlussendlich darin, eine Botschaft 
so eindrucksvoll wie möglich zu vermitteln, sodass das Publikum die gewünschte Aussage annimmt, 
sie bestenfalls akzeptiert und dann resultierend handelt.

\subsection{Historischer Hintergrund}
Ursprünglich entstand die Rhetorik etwa in der Mitte des 5. Jahrhunderts auf Sizilien, genau genommen im 
antiken Griechenland, bei dem die Griechen als erstes eine Systematik für die Kunst der öffentlichen 
Rede entwickelten. In Griechenland wurde diese Art des Sprachgebrauchs häufig und vielseitig eingesetzt. 
So konnten Konflikte statt eines Kampfes oft vor Gericht ausgetragen werden und kam es darauf an, wer die 
bessere Redekunst besaß und den Richter überzeugen konnte. \\Da in diesem Zeitalter noch nicht jeder 
ausgereichnet gebildet war, um Rhetorik anzuwenden, wendete sich die Gesellschaft an Redelehrer. Ein 
bedeutender Lehrer war zum Beispiel Korax. Er und sein Lehrling Teisias galten als Gründer der 
modernen Rhetorik. Rhetorik bezieht sich auf die kommunikativen Fähigkeiten der 
Überzeugung - wie in der Politik. Zu Beginn des Mittelalters war die Rhetorik so bedeutend, dass 
ohne den Erwerb der Grundlagen, das Studieren an einer Universität nicht gestattet war. Da außerdem 
im Mittelalter viel Wissen über die griechische und römische Rhetorik verloren ging, kam es zu einem 
geistigen Rückfall. \\Zur Zeit der Renaissance um ca. 1453 flohen viele Menschen aufgrund von Kriegen, 
Unsicherheiten und dem Fall von Konstantinopel (Eroberung durch das römische Reich) Richtung Westen. 
Da vor allem Gelehrte vor dem Krieg flohen, brachten sie zahlreiche alte Schriften mit, welche viel 
altes verlorenes Wissen der Antike enthielten. So konnte sich dieses wieder 
angeeignet, modernisiert und weiterentwickelt werden. Doch im 19. Jahrhundert erfuhr die Rhetorik als 
Lehrfach einen weiteren Abwärtstrend, denn es gab bekannte Gegner, wie zum Beispiel Immanuel Kant 
und Goethe, welcher das Lehrfach als „Schule des Verstellens“ bezeichnete. Ihrer Meinung nach sei die 
Redekunst allein dazu da, die Schwächen des Gegners auszunutzen, anstatt selbst Argumente zu liefern.  
\\Im 20. Jahrhundert lebte die Rhetorik wieder auf und erholte sich von ihrem schlechten Ruf. 
Viele Theoretiker haben die Kunst der Rhetorik studiert und sie aus der Dunkelkammer geholt. 
Rhetorik ist heute ein fester Bestandteil unserer schnelllebigen Gesellschaft. Wir müssen 
erfolgreich überzeugen, beeinflussen, gewinnen und kommunizieren, sowohl im täglichen Berufsleben 
als auch im Privatleben. Als man erkannte, wie wichtig und sinnvoll sie ist, entwickelte man 
zusätzlich eine \textit{„Gebrauchs-Rhetorik“}. In der Literaturwissenschaft ist sie wieder ein fester 
Bestandteil. 

\subsection{Säulen der Rhetorik}
\label{sec: Säulen der Rhetorik}
Insgesamt gibt es drei verschiedene Säulen der Rhetorik. Parallel zu verwenden, ist die Begrifflichkeit 
des \textit{rhetorischen Dreiecks}. In dieser Arbeit wird dieses als die drei Säulen der Rhetorik beschrieben. 
Aristoteles stellte die drei Säulen auf.
\\Früher hing Rhetorik nur mit Logos zusammen, was abgeleitet aus dem Begriff \textit{„Logik“} stammt. 
Allein konnte Logos nicht als Begriff für Rhetorik stehen, sodass Ethos und Pathos als weitere 
Anhaltspunkte im Bezug auf Rhetork zu verwenden sind. 
\\Wer dementsprechend als Ziel hat zu überzeugen, sollte nicht nur Fakten nennen, sondern gleichzeitig  
Daten (Logos) hinzufügen und glaubwürdig (Pathos) und charakterfest (Ethos) wirken. Wenn der 
Autor diese drei Säulen \bzw Anhaltspunkte verwendet, gewinnt er definitiv an Überzeugungskraft 
beim Publikum.

\subsubsection{Ethos}
Ethos ist der griechische Begriff für \textit{„ethisches Bewusstsein, Sitte und Brauch“}. Wie 
Aristoteles sagte, \textit{„Als Redner überzeugen Sie durch moralischen Charakter und einer so dargebotenen 
Rede, sodass Sie sich Vertrauen verdienen.“}. Ethos zielt in der Rhetorik darauf ab, das Vertrauen 
und die Überzeugungskraft aufzubauen, indem der Autor zum Beispiel seine Kompetenz demonstriert oder 
seine moralische Autorität. Wenn er einen guten Charakter zeigt und seriös dem Publikum gegenübersteht, 
stärkt er die Wirkung sein Ethos und die Personen neigen eher dazu dem Redner oder Autor zu vertrauen. 
Sobald das Publikum dem Autor vertraut, akzeptieren sie die genannten Argumente viel zeitiger. Ein 
Beispiel dafür ist, wenn ein Arzt seine medizinischen Ausbildungen zeigt, um zu betonen wie gut er ist, 
indem was er tut und die Patienten auf seine Ratschläge vertrauen und hören sollen, bedingt desse, dass er sich in dieser Thematik hervorragend auskennt. Um beim Publikum überzeugen zu können, müssen  einige Punkte eingehalten werden. Die Redner müssen zeigen wie qualifiziert sie sind und dass das Publikum sich auf 
diesen verlassen kann. Dabei helfen auch einige persönliche Informationen und ist es wichtig 
Anhaltspunkte zu bringen, welche sie von der Menge abhebt, damit sie im Kopf der Zuhörer bleiben. 

\subsubsection{Pathos}
Vor einiger Zeit waren die Menschen der Auffassung, dass eine gute Argumentation zur Überzeugung völlig ausreichend ist. Das Pathos hingegen arbeitet primär mit Emotionen und mit den gesammelten  Erfahrungen die zur Erreichung des Ziels beitragen, um das Publikum regelrecht mitzureißen. Ein dafür häufig genutztes rhetorisches Mittel ist die Metapher. \\Beim Publikum kann eine gut dargestellte Geschichte, in welcher der Autor von eigenen Gefühlen erzählt, ob positiv oder negativ, gut überzeugend sein. Das Erzählte bleibt dadurch besser im Kopf und regt zusätzlich zum Nachdenken an. Durch die hervorgerufenen Emotionen, des vom Autor angewandten Pathos, werden Entscheidungen primär gesteuert. \\Hierzu ein Beispiel, wie Pathos das Publikum berührt: \textit{“Die hilflosen 
Blicke der Tiere im Tierheim rühren an unser Mitgefühl. Gemeinsam können wir ihnen Hoffnung und ein neues Zuhause schenken!“}. Bedingt desse, dass Pathos nur durch Emotionen eine Verbindung zum Publikum herstellen kann, bringt es in diesem Fall nichts, mit Fakten oder Können zu überzeugen. Besser ist die Verwendung von Metaphern, denn diese sorgen für eine bildliche Sprache und das menschliche Gehirn kann so mit Bildern arbeiten. Ebenfalls helfen Schlagwörter und kurze Sätze, um freier und authentischer zu wirken.

\subsubsection{Logos}
\label{sec: Logos}
Logos ist die letzte Säule der Rhetorik und spiegelt Fakten und Sachlichkeit wieder. 
\\Aristoteles präferierte ebenfalls Zahlen, Daten und Fakten für eine gute Argumentation zu nutzen.  Ein guter Redebeitrag basiert nun mal auf Fakten, was dafür sorgt, dass der Status als Experte mehr untermauert wird, welches mit Hilfe von Statistiken, Beweisen oder Zitate hervorrufen kann. Auch wenn Logos alleine bei einem Publikum nicht überzeugende Kraft hat, bedingt der Trockenheit, kann bei einem Redebeitrag darauf nicht verzichten werden, da es die sachliche Basis einer Rede ist. Sobald dieser Bestandteil nicht mehr eingebracht wird, kann der Redner seine Glaubwürdigkeit verlieren. Beispielsweise Statistiken belegen - \textit{„gesündere Ernährung beugt Krankheiten vor und fördert die Gesundheit. Gesunde Ernährung ist daher eine logische Entscheidung für ein gesünderes Leben.“}. Um Logos richtig anzuwenden, kann der Sprecher nicht nur Fakten aufzählen, sondern strukturiert vorgehen. Denn bei einer Nichteinhaltung, können die Zuhörenden den möglichen Inhalt nicht verfolgen und ein unseriöses Auftreten kann passieren. Aus diesem Grund ist es wichtig, sich an eine sachliche Argumentation zu halten, denn Emotionen haben nicht mit Logos gemein.
\\Um schlussendlich einen erfolgreichen und sicheren Eindruck zu machen, werden die drei Säulen in Kombination benötigt. Natürlich ist die Argumentation des Sprechers abhängig des zu erreichenden Ziels und der vorzusprechenden Menge.

\subsection{Was sind rhetorische Mittel?}
\label{sec: Was sind rhetorische Mittel?}
Rhetorische Mittel sind eine Unterkategorie von Stilmitteln, weil sie eher in gesprochener Sprache verwendet werden und im Gegensatz zu der anderen Unterkategorie sprachliche Mittel, welche generell alle Ausdrucksmittel sind, die im schriftlichen verwendet werden. 
\\Häufig genutzt, als sprachliche Mittel sind Metaphern, Alliteration, Parataxen und Ironie. Sie werden in erster Linie eingesetzt, um die gesprochen Sprache ansprechend zu gestalten und zu überzeugen, interessanter und emotionaler zu wirken. Beispielsweise bei Präsentationen oder auch ganz besonders politische Reden. Dies verbindet sie mit der Rhetorik, wodurch sie sozusagen die sprachlichen Werkzeuge der Rhetorik sind.

\subsubsection{Rhetorische Mittel in einer Rede und ihre Wirkung}
\label{sec: Rhetorische Mittel in einer Rede und ihre Wirkung}
Ein rhetorische Mittel ist die Alliteration. Unter dieser versteht man aneinander gereihte Begriffe mit demselben Anfangslaut. Sie sorgen für mehr Aufmerksamkeit vom Publikum. Damit sie eine starke Wirkung erzielt, ist eine klare und betonte Aussprache notwendig. Alliterationen eignen sich besonders für den Anfang der Rede um die Aufmerksamkeit der Zuhörerschaft zu gewinnen. Ein Beispiel ist die Redewendung \textit{„durch dick und dünn“}.
\\Zunächst kann häufig auf die Anapher zurückgegriffen werden, die durch Wiederholungen am Satzanfang eingesetzt wird. Ziel dabei ist immer, das Gesagte im Gedächtnis des Zuhörers verweilen zu lassen, um die wichtigsten Sachverhalte in Erinnerung behalten zu können, wie zum Beispiel mit dem Sprichwort \textit{"Wer kämpft, kann verlieren; wer nicht kämpft, hat schon verloren"}.
\\Oft findet die Hyperbel im Sprachgebrauch eine Verwendung, weil sie als eine überzogene Darstellung eines Sachverhaltes, zu einer emotionalen, lebendigen und bildlichen Rede führen kann. Vermehrt auch bildlicher, aufgrund der meist im Zusammenhang mit einer Metapher stehenden und der Ausdruckskraft der Stilmittel verstärkenden Einbring der Wirkung. Oftmals kann sie für Kritiken und Aussagen mit besonderem Nachdruck einen Einsatz finden, wie unter anderem \textit{"Jemanden zum Fressen gern haben"}.
\\Eines der bekanntesten rhetorischen Mittel, ist die Metapher, welche gleichermaßen als eine bildliche Darstellung eines Vergangs gesehen werden kann. Sie ersetzt lange Erklärungen, was einen Langeweile-Effekt beim Publikum erreichen kann und das Interesse im Laufe der Rede schwindet. Meist findet die Metapher Anwendung im Hauptteil einer Präsentation, die je nach Begriffserklärung die Aussagen anschaulicher rüberbringen kann. Ein Klassiker ist zum Beispiel die Redewendung \textit{eine Nadel im Heuhaufen suchen}.
\\Die rhetorische Frage, als eine zu stellende Frage, bei der keine Antwort erwartet wird. Die Funktion hinter dieser besteht darin, die Zuhörer unauffällig zu manipulieren, um eine gezielte Antwort im Kopf zu erhalten. Das Publikum stimmt automatisch zu, da der Redner den Zuschauern die offensichtliche Antwort nahelegt. Des Weiteren regt eine Frage zum Nachdenken an, weshalb die Zuhörer länger beschäftigt werden können. Ein Beispiel kann die Fragestellung \textit{"Machen wir nicht alle Fehler?"} sein.
\\Auch der Klimax, als eine Steigerung vom Kleinsten zum Größten, wird oft in Sachverhalten genutzt. Grund dafür ist, die Spannung aufrecht zu erhalten und von dem Zuzuhörenden die vollste Aufmerksamkeit zu erhalten. Vergleichbar ist der Klimax mit einer Erörterung, beo der erst gegen die Opposition und dann die eigene Argumentation belegt wird, wobei von dem schwächsten zum stärksten Glied gegangen wird. So bleibt das Gesagt im Kopf und der Spracher kann seinen Handlungsvorschalg eindrucksvoll präsentieren. So auch eines der einprägsamsten Klimaxe: \textit{Er kahm, sah und siegte}.

\subsection{Rhetorik im Alltag}
\label{sec: Rhetorik im Alltag}
Nicht nur heute, sondern auch vor einigen Jahrhunderten war die Rhetorik präsent. Dabei kann es sich im privaten Bereich beispielsweise um einen Small Talk (kurze Unterhaltung) oder Diskussion handeln oder im geschäftlichen Bereich, um ein fundiertes Grundwissen für Führungspositionen oder in der Politik.
\\So kann in Konfliktsituationen eine durchdachte Anwendung gewinnbringend sein. Auch bei Bewerbungsgesprächen darf eine gut angewandte Rhetorik nicht fehlen, die bei erfolgreicher verwendeter Technik der Fähigkeiten- und Qualifikationsdarstellung den Bewerbungsprozess beeinflussen können. Auch in Diskussionen werden rhetorische Techniken genutzt, um Argumente überzeugender zu veranschaulichen und das Publikum zu beeinflussen, in alltäglichen Situationen, wie in informellen Gesprächen und Alltagssituationen können rhetorische Strategien verwendet werden, um andere von einer Idee zu überzeugen oder Zustimmung zu erhalten. Aus dem gesamten Abschnitt erfolgt die Schlussfolgerung, dass jeder Rhetorik beherrschen sollte. 
\\Wir selber werden von Rhetorik im Alltag beeinflusst, wie beispielsweise von politischen Reden oder auch in Unternehmens-Präsentationen, wo die Führungskräfte Rhetorik verwenden, um die Mitarbeiter zu motivieren. Auch in verschiedenster Werbung, sei es ein Video oder ein Plakat, wird die rhetorische Strategie verwendet, wie auch in der Geschichte und noch heute die Rhetorik bei Gerichtsverhandlungen eine Rolle, um die Argumente überzeugend darzulegen spielt.

\subsection{Die Bedeutung rhetorischer Fähigkeiten für die Kommunikation}
\label{sec: Die Bedeutung rhetorischer Fähigkeiten für die Kommunikaion}
Diese Kunst ist in vielen Lebensbereichen anwendbar und nützlich wie bei Präsentationen oder Vorstellungsgesprächen. Mit einer gut entwickelten Rhetorik und bei Verwendung der 3 Säulen, siehe \ref{sec: Säulen der Rhetorik} \nameref{sec: Säulen der Rhetorik} können sie ihre Mitmenschen von ihren Worten überzeugen und sie in ihrem Gedächtnis festigen. Doch es hilft nicht nur, um Menschen zu überzeugen, sondern es hilft auch, das eigene Selbstbewusstsein zu stärken, denn um so leichter es einem fällt, umso einfacher ist es selbstbewusster aufzutreten und unter anderem auch eine gute Grundlage für Diskussionen zu bilden und verleiht dadurch eine gute sprachliche Flexibilität.

\subsection{Rhetorikanwendung in politischen Reden}
\label{sec:Rhetorikanwendung in politischen Reden}
Wie bereits erwähnt, finden die rhetorichen Mittel in der Politik eine größe und einflussreiche Anwendung. Diese werden als schriftliche Rede im Vorfeld erabeitet und richten sich immer an die Zuhörerschaft. In den meisten Fällen gibt eine bestimmte Situation das Thema vor, woran man die Rede aufbauen kann. Wichtig ist, dass die Erwartungen der Empfänger, die Wertvorstellung oder den Wissensstand bedeutsam in die Rede einfließen lässt. \\Ich als Redner habe ein Ziel, die Zuhörer von etwas zu unterrichten und sie zum nachdenken oder handeln zu motivieren. Ich kann das Publikum mit verbalen und nonverbalen Botschaften durch eine äußerliche Inszenierung überzeugen. Hierbei muss ich auf den Inhalt der Formulierung achten und darf den Wissensstand meiner Zuhörer nicht außer Acht lassen.