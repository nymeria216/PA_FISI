% !TEX root = ../Projektdokumentation.tex
\section{Entwurfsphase} 
\label{sec:Entwurfsphase}

\subsection{Zielplattform}
\label{sec:Zielplattform}
Die zu resultierende Zielplattform definiert sich über die Benutzerfreundlichkeit, die Sicherheit und dem Fokus auf der Interaktion 
mit \acs*{OIDC} und \acs*{LDAP}, welche zukünftig für das Projekt vorgesehen sind, sowie der Implementierung vom Open-Source, dem Arbeiten mit MFA, 
der Skalierbarkeit und den Erfahrungswerten von anderen Entwicklern mit dem entsprechenden Produkt. Das Resultat wird im Kapitel 
\ref{sec:Authentifizierungs-Tool}: \nameref{sec:Authentifizierungs-Tool} der dargestellten~\nameref{tab:Nutzwert} sichtbar.


\subsection{Authentifizierungs-Tool}
\label{sec:Authentifizierungs-Tool}
Anhand der Entscheidungsmatrix in Tabelle~\ref{tab:Nutzwert} wurde Authentik ausgewählt. 
\\Die im Kapitel~\ref{sec:Zielplattform}: \nameref{sec:Zielplattform} erwähnten Eigenschaften, tragen zur 
Entscheidungsfindung bei, sodass sich anhand der gegebenen~\nameref{tab:Nutzwert} Authentik herauskristallisierte.
\\\tabelle{Entscheidungsmatrix}{tab:Nutzwert}{Nutzwert.tex}
\\Die Gewichtung bildet sich durch den Gesamtwert von 100 Wertungspunkten mit einem Maximalwert von 25 und Mindestwert von 10. 
Grund dafür sind die unterschiedlichen Eigenschaften, welche in der Entwicklung und bei der Auswahl der 
Authentifizierungsmethode jeweils verschiedene Rollen spielen und folglich daraus evaluiert werden. Nach der Zuordnung 
und Addierung der Punkte bei den vier Authentifizierungs-Tools, wird die Gesamtheit aller Punkt eines Produktes durch den 
Wert 100 dividiert und das Endergebnis ausgerechnet. Dabei hat die Einhaltung der Sicherheit Priorität und erhält den 
Maximalwert von 25 Punkten, aufgrund dessen, dass Dritten der Zugriff auf die jeweiligen Dienste mit kunden- und 
firmeninternen Daten der Cloud-Infrastruktur verweigert werden sollte. Die \acs*{MFA} wird mit 20 Punkten bewertet, weil 
dass das Ziel des Projektes ist. Die Benutzerfreundlichkeit, das Implementieren mit Open-Source und die Skalierbarkeit 
erhalten 15 Punkte, weil jeder User schnell und einfach auf den jeweiligen Dienst zugreifen muss. 
Für dieses Projekt ist es des Weiteren wichtig, ein Tool zu implementieren, was den zeitlichen Rahmen nicht überschreitet und die 
Gesamtheit der Einführung zu komplex gestaltet. Die Skalierbarkeit hat für das Projekt eine durchschnittliche Relevanz, da die 
Möglichkeit bestehen soll, Dienste hinzuzufügen oder rauszunehmen. Am wenigsten bedeutend sind die Erfahrungswerte, welche 
gleichermaßen nicht zu unterschätzen sind, weil eine Community über das Produkt bei der Entwicklung unterstützend wirkend kann.
\\Dabei erhält Authentik den größten Nutzwert mit 17,05 Punkten und schneidet vor Authelia, Microsoft Azure AD und Sitecar am besten ab. 
Aus diesem Grund wird sich für Authentik anstatt für Sitecar entschieden, da besonders die Implementierungsdauer und -komplexität 
bei Sitecar den Rahmen des Projektes sprengen würde.

\subsection{Geschäftslogik}
\label{sec:Geschaeftslogik}
Im \nameref{sec:Anhang} befindet sich eine genaue Darstellung der \nameref{app:Cloud-Infrastruktur} und des \nameref{app:UseCase}s, 
sowie das \nameref{app:Sequenzdiagramm} zur besseren Visualisierung. 
\\Die Implementierung von Authentik erleichert den Entwicklern den Arbeitsfluss durch das einmalige Anmelden bei allen Diensten.

\subsection{Maßnahmen zur Qualitätssicherung}
\label{sec:Qualitaetssicherung}

\subsubsection{Produktorientierte Maßnahmen}
\label{sec:ProduktorientierteMaßnahmen}

\subsubsection{Prozessorientierte Maßnahmen}
\label{sec:ProzessorientierteMaßnahmen}
Durch die kontinuierliche Anwendung des Qualitätsmanagementsystems gemäß ISO 9001 im Qualitätsmanagement erfolgt nach der Planung 
eine sorgfältige Überprüfung jedes Schrittes auf Richtigkeit. Bei festgestellten Fehlern wird nach alternativen Wegen zur 
Zielerreichung gesucht, um schließlich erfolgreich umzusetzen. Dieser Prozess folgt dem Plan-Do-Check-Act-Zyklus.
\\Als Online-Dienstleister ist es laut den Vorschriften des Bundesamts für Sicherheit in der Informationstechnik (BSI) 
zwingend erforderlich, Daten und Geräte durch eine Mehrfaktor-Authentifizierung (\acs*{MFA}) zu schützen, bei der der Login durch 
die Verwendung eines Passworts und eines weiteren Faktors abgesichert wird. Authentik, als Identitätsprovider, nutzt ein 
Informationssicherheitssystem, das auf etablierten Standards wie BSI100-2 und ISO27001 basiert.