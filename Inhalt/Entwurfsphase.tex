% !TEX root = ../Projektdokumentation.tex
\section{Entwurfsphase} 
\label{sec:Entwurfsphase}

\subsection{Zielplattform}
\label{sec:Zielplattform}
Die zu resultierende Zielplattform definiert sich über die Benutzerfreundlichkeit, Sicherheit, Implementierung, 
Dokumentation und Skalierbarkeit und entscheidet sich durch die im Kapitel~\ref{sec:Architekturdesign}: \nameref{sec:Architekturdesign} 
dargestellte Nutzwertanalyse.


\subsection{Authentifizierungs-Tool}
\label{sec:Authentifizierungs-Tool}
Anhand der Entscheidungsmatrix in Tabelle~\ref{tab:Entscheidungsmatrix} wurde Authelia ausgewählt. 
\\Die in Kapitel~\ref{sec:Zielplattform}: \nameref{sec:Zielplattform} erwähnten Eigenschaften tragen zur Entscheidungsfindung bei, sodass 
sich anhand der gegebenen~\nameref{tab:Entscheidungsmatrix} Authelia herauskristallisierte.
\\\tabelle{Entscheidungsmatrix}{tab:Entscheidungsmatrix}{Nutzwert.tex}
\\Die Gewichtung hat einen Gesamtwert von 100 Wertungspunkten mit einem Maximalwert von 25 und Mindestwert von 15. Grund dafür sind die 
unterschiedlichen Eigenschaften, welche in der Entwicklung und bei der Auswahl der Authentifizierungsmethode jeweils verschiedene Rollen 
spielen und resultierend daraus variierend evaluiert werden. Nach der Zuordnung und Addierung der Punkte bei den drei Authentifizierungs-Tools, 
wird die Gesamtheit aller durch den Wert 100 dividiert und das Endergebnis ausgerechnet. Dabei hat die Einhaltung der Sicherheit Priorität und 
erhält den Maximalwert von 25 Punkten, bedingt dessen, dass Dritten der Zugriff auf die jeweiligen Dienste mit kunden- und firmeninternen Daten 
der Cloud-Infrastruktur verweigert werden sollte. 
Die Benutzerfreundlichkeit, Skalierbarkeit und Implementierung werden mit 20 Punkten bewertet, aufgrund der Tatsache, dass jeder User schnell und 
einfach auf den jeweiligen Dienst zugreifen muss. Für dieses Projekt ist es des Weiteren wichtig, ein Tool zu implementieren, was den 
zeitlichen Rahmen nicht überschreitet und die Gesamtheit der Einführung zu komplex gestaltet. Die Skalierbarkeit hat für das Projekt eine 
Relevanz, weil die Möglichkeit besteht, Dienste hinzuzufügen oder rauszunehmen. Am wenigsten bedeutend ist die Dokumentation, welche 
gleichermaßen nicht zu unterschätzen ist, da eine gut geschriebene Dokumentation des Herstellers die Entwicklung enorm beeinflusst und erleichert.

\subsection{Geschäftslogik}
\label{sec:Geschaeftslogik}

\begin{itemize}
	\item Modellierung und Beschreibung der wichtigsten (!) Bereiche der Geschäftslogik (\zB mit Kom\-po\-nen\-ten-, Klassen-, Sequenz-, Datenflussdiagramm, Programmablaufplan, Struktogramm, \ac{EPK}).
	\item Wie wird die erstellte Anwendung in den Arbeitsfluss des Unternehmens integriert?
\end{itemize}

\paragraph{Beispiel}
Ein Klassendiagramm, welches die Klassen der Anwendung und deren Beziehungen untereinander darstellt kann im \Anhang{app:Klassendiagramm} eingesehen werden.

\Abbildung{Modulimport} zeigt den grundsätzlichen Programmablauf beim Einlesen eines Moduls als \ac{EPK}.
\begin{figure}[htb]
\centering
\includegraphicsKeepAspectRatio{modulimport.pdf}{0.9}
\caption{Prozess des Einlesens eines Moduls}
\label{fig:Modulimport}
\end{figure}


\subsection{Maßnahmen zur Qualitätssicherung}
\label{sec:Qualitaetssicherung}
\begin{itemize}
	\item Welche Maßnahmen werden ergriffen, um die Qualität des Projektergebnisses (siehe Kapitel~\ref{sec:Qualitaetsanforderungen}: \nameref{sec:Qualitaetsanforderungen}) zu sichern (\zB automatische Tests, Anwendertests)?
	\item \Ggfs Definition von Testfällen und deren Durchführung (durch Programme/Benutzer).
\end{itemize}


\subsection{Pflichtenheft/Datenverarbeitungskonzept}
\label{sec:Pflichtenheft}
\begin{itemize}
	\item Auszüge aus dem Pflichtenheft/Datenverarbeitungskonzept, wenn es im Rahmen des Projekts erstellt wurde.
\end{itemize}

\paragraph{Beispiel}
Ein Beispiel für das auf dem Lastenheft (siehe Kapitel~\ref{sec:Lastenheft}: \nameref{sec:Lastenheft}) aufbauende Pflichtenheft ist im \Anhang{app:Pflichtenheft} zu finden.
