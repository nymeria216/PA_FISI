% !TEX root = ../Projektdokumentation.tex
\section{Projektdurchführung} 
\label{sec:Projektdurchführung}

\subsection{Vorbereitung der Entwicklungsumgebung}
\label{sec:Vorbereitung der Entwicklungsumgebung}
Die Voraussetzung zur Implementierung von Authentik ist, dass Docker und Docker Compose auf dem im Kapitel \ref{sec:Sachmittelplanung} 
\nameref{sec:Sachmittelplanung} erwähnten Notebook vorinstalliert sind und einen Zugang zum Internet verfügen. Des Weiteren sollten die zu 
benötigenden Hard- und Softwarekomponenten, welche ebenfalls in der \nameref{sec:Sachmittelplanung} auf der Seit \pageref{sec:Sachmittelplanung} 
gelistet sind, funktionierend sein. Wichtig für die Durchführung ist nicht nur Docker mit dem Notebook selbst, sondern auch der 
Besitz einer Domain oder Subdomain, welche bei der Deloitte Wirtschaftsprüfungsgesellschaft GmbH, die \textbf{tal.deloitte.de} ist. 
Diese muss entweder mit einem A- oder CNAME-Record versehen sein. Um die Domain-Names übersichtlich verwalten zu können, 
empfiehlt sich ein Reverse Proxy, in diesem Fall wird es auf den NGinx Proxy Manager zurückzuführen sein. Und zu guter letzt einen SMTP 
E-Mail Server zur beispielsweise Zurücksetzung eines Passworts. Für das Testen der kommenden Schritte im Kapitel 
\ref{sec:Konfiguration des zweiten Faktors} \nameref{sec:Konfiguration des zweiten Faktors} ab der Seite 
\pageref{sec:Konfiguration des zweiten Faktors} wird das iPhone 12, erwähnt in der \nameref{sec:Sachmittelplanung}, verwendet. 
Dieses mobile Endgerät ist ein Firmentelefon und beinhaltet zusätzlich den Microsoft Authenticator, um sich 
gewissermaßen remote mit dem VPN des Firmennetzwerkes verbinden zu können. Aus diesem Grund muss kein weiteres Active Directory 
installiert und eingerichtet werden.

\subsection{Auswahl einer MFA-Lösung}
\label{sec:Auswahl einer MFA-Lösung}
Bei Authentik werden die drei Optionen \textbf{WebAuthn Authenticator Setup Stage}, \textbf{Static Authenticator Stage} und 
\textbf{\acs{TOTP} Authenticator Setup Stage} angeboten. 
\\ Das Ergebnis der Tabelle \ref*{tab:Nutzwertanalyse} \nameref{tab:Nutzwertanalyse} 
\tabelle{Nutzwertanalyse zur MFA-Lösung}{tab:Nutzwertanalyse}{Nutzwertanalyse_MFA.tex}
% Bei der Deloitte Wirtschaftsprüfungsgesellschaft GmbH wird das \acs{TOTP}-Verfahren (Eingabe eines Einmalpassworts) empfohlen. 
% Zu Beginn geben, in dem Fall des Projektes, die Entwickler :innnen das Passwort und den Nutzernamen ein. 
% Bei einer erfolgreichen Eingabe haben die Entwickler: innen die Möglichkeit sich weiter über den zeitbasierten Einmalcode, generiert im Microsoft 
% Authenticator, einzugeben. Vorteilhaft ist der schon vorhandene Authenticator, welcher nicht zusätzlich installiert und konfiguriert 
% werden muss. Da jede zu benutzende Person bei der Deloitte das Firmentelefon den Microsoft Authenticator schon vorinstalliert 
% bei der Einstellung dieser erhält, entfällt die Konfiguration, sodass nur noch der QR-Code des jeweiligen Dienstes gescannt werden muss.


\subsection{Erstellung der docker-compose.yml, .env}
\label{sec:Erstellung der docker-compose.yml, .env}
Um Authentik über eine ''\nameref{app:docker-compose.yml}''-Datei im \nameref{sec:Anhang} des Kapitels \ref{app:docker-compose.yml} 
zum Laufen zu bringen, ist es erforderlich, dass nicht nur der Authelia-Server, sondern auch der Authelia-Worker, 
Redis und PostgreSQL gleichzeitig erstellt werden. PostgreSQL ist eine objektrelationale Datenbank und unterstützt 
die Erweiter- und Skalierbarkeit der Cloud-Infrastruktur. Redis hingegen ist ein In-Memory-Datenspeicher, der als 
schneller Datenspeicher dient. Die ''\nameref{app:dotenv}''-Datei im Kapitel \ref{app:dotenv} \nameref{app:dotenv} 
enthält die individuellen Umgebungsvariablen für die Zugangsdaten der Nutzer: innen. 
\\Beide Dateien werden in einem Ordner ''Authentik'' auf der Instanz \textbf{''tal\_cloud\_infra''} durch docker erstellt.

\subsection{Installation und Konfiguration von Authentik}
\label{sec:Installation und Konfiguration von Authentik}
% Nachdem Authentik in einem Docker-Container läuft, erfolgt nun der Zugriff über den Nginx. Dabei wird vor dem Zugriff 
% eine Demonstrations-Domain auf dem Nginx Proxy Manager erstellt, die Authentik vorerst darstellen soll. 
% So wird verhindert, dass bei einer Konfiguration von Authentik, kein Service kaputt gehen kann und aufgrund dessen der 
% Service geschützt bleibt.


\subsection{Integration mit Cloud-Diensten}
\label{sec:Integration mit Cloud-Diensten}

\subsection{Konfiguration des zweiten Faktors}
\label{sec:Konfiguration des zweiten Faktors}