% !TEX root = ../Projektdokumentation.tex
\section{Projektdurchführung} 
\label{sec:Projektdurchführung}

\subsection{Vorbereitung der Entwicklungsumgebung}
\label{sec:Vorbereitung der Entwicklungsumgebung}
Die Voraussetzung zur Implementierung von Authentik ist, dass Docker und Docker Compose auf allen Linux-Umgebungen der 
Instanzen installiert sind, was schon der Fall ist, und über einen Zugang zum Internet verfügen. Die Installation ist erforderlich, da die 
Conatinerisierung der \acs{MFA}-Lösung nicht wahrgenommen werdenn kann. Des Weiteren sollten die zu 
benötigenden Hard- und Softwarekomponenten, welche ebenfalls in der \nameref{sec:Sachmittelplanung} auf der Seite \pageref{sec:Sachmittelplanung} 
gelistet sind, funktional sein. Wichtig für die Durchführung ist der Besitz einer Domain oder Subdomain, welche bei der Deloitte 
Wirtschaftsprüfungsgesellschaft GmbH, aus Datenschutz-Gründen die \textbf{domain.de} ist. 
Die eigentliche Domain der Deloitte wird mit der \textit{example.domain.de} und später der \textit{auth.example.domain.de} ersetzt.
Diese muss entweder mit einem \acs{A}- oder \acs{CNAME}-Record versehen sein. Mit einem \acs{A}-Record wird eine Domain oder Subdomain an eine 
IP-Adresse weitergeleitet und gehört zu den wichtigsten Arten von \acs{DNS}-Einträgen. Der \acs{CNAME}-Record hingegen definiert einen Alias für einen 
anderen \acs{DNS}-Record und ist im Grunde eine Alternativbezeichnung einer Domain und bedeutet in deutsch so viel wie 
autorisierter Name. Um die Domain-Namen übersichtlich verwalten zu können, Anschließend wird ein E-Mail Server benötigt, um beispielsweise das 
Zurücksetzung eines Passworts zu ermöglichen. Für das Testen der kommenden Schritte im Kapitel 
\ref{sec:Einrichtung des zweiten Faktors} \nameref{sec:Einrichtung des zweiten Faktors} ab der Seite 
\pageref{sec:Einrichtung des zweiten Faktors} wird das iPhone 12, erwähnt in der \nameref{sec:Sachmittelplanung}, verwendet. 
Dieses mobile Endgerät ist ein Firmentelefon und beinhaltet den installierten und schon eingerichteten Microsoft Authenticator. 

\subsection{Auswahl einer MFA-Lösung}
\label{sec:Auswahl einer MFA-Lösung}
Bei Authentik werden die drei Optionen \textbf{WebAuthn Authenticator Setup Stage}, \textbf{Static Authenticator Stage} und 
\textbf{\acs{TOTP} Authenticator Setup Stage} angeboten. 
\\Das Ergebnis der Tabelle \ref*{tab:Nutzwertanalyse} \nameref{tab:Nutzwertanalyse} ist nicht eindeutig zuordbar, bedingt dessen, dass 
die Optionen WebAuthn Authenticator Setup Stage und \acs{TOTP} Authenticator Setup Stage um 0,05 Wertungspunkte auseinander liegen und 
die Entscheidung des finalen Ergebnisses nicht anhand der \nameref{tab:Nutzwertanalyse} manifestierend festzulegen ist.
\tabelle{Nutzwertanalyse zur MFA-Lösung}{tab:Nutzwertanalyse}{Nutzwertanalyse_MFA.tex}
Die Entscheidung wurde maßgeblich durch die Tatsache beeinflusst, dass die Deloitte Wirtschaftsprüfungsgesellschaft GmbH das 
\acs{TOTP}-Verfahren für sämtliche verfügbaren Dienste einsetzt. Dieses Verfahren erfordert die Eingabe eines einmaligen Passworts 
über einen Authenticator, wie es beim iPhone 12 der Fall ist. Daraufhin fiel die Entscheidung auf das \textbf{\acs{TOTP}-Verfahren}, welches 
hingegen zum WebAuth Authenticator Setup Stage weniger Sicherheit aber weniger Implementierungsaufwand erfordert und kein weiteres 
Gerät oder eine Software notwendig ist. 
In dem \acs{TOTP} Authenticator Setup Stage geben alle Benutzer zu Beginn ihren Nutzernamen und das zugehörige Passwort ein. 
Nach einer erfolgreichen Anmeldung, werden die Benutzer anschließend weitergeleitet, um einen 6-stelligen Zahlencode einzugeben.
Dieser Code ist in dem Microsoft Authenticator für 30 Sekunden sichtbar. Wenn die Benutzer sich noch nicht über Authentik angemeldet hatten, 
werden diese nach der Anmeldung aufgefordert, einen QR-Code mit dem Microsoft Authenticator abzuscannen, wodurch der 6-stellige Code generiert wird. 
Dieser ist daraufhin, wie oben beschrieben, bei jeder Anmeldung einzugeben.

\subsection{Erstellung der docker-compose.yml und .env}
\label{sec:Erstellung der docker-compose.yml .env}
Um Authentik über eine \textit{\nameref{app:docker-compose.yml}}-Datei lauffähig zu machen, müssen der Authentik-Server, der Authentik-Worker, 
Redis und PostgreSQL gleichzeitig erstellt werden, ws durch die Dcker-Containersisierung geschieht. 
PostgreSQL ist eine objektrelationale Datenbank und unterstützt 
die Erweiter- und Skalierbarkeit der Cloud-Infrastruktur. Redis hingegen ist ein In-Memory-Datenspeicher, der als 
schneller Datenspeicher dient. Die ''\nameref{app:dotenv}''-Datei im Kapitel \ref{app:dotenv} \nameref{app:dotenv} 
enthält die individuellen Umgebungsvariablen für die Zugangsdaten der Nutzern. 
% \\Beide Dateien werden in einem Ordner ''Authentik'' auf der Instanz \textbf{''tal\_cloud\_infra''} durch docker erstellt. 
Eine Darstellung der Instanzen befindet sich im Kapitel \ref{app:Cloud-Infrastruktur} \nameref{app:Cloud-Infrastruktur}. 
Die ''\nameref{app:docker-compose.yml}''-Datei wurde über \cite{Composerize}, einer 
Applikation aus dem Internet entnommen, welche öffentlich zugänglich ist. Auch die ''\nameref{app:dotenv}''-Datei ist öffentlich über die 
Webpage von Authentik einsehbar. Mittels Composerize ist es möglich anstatt eines Docker-Befehls, eine docker-compose.yml-Datei zu erstellen.
\\Die einzigen in dieser Datei vorzunehmenden Änderungen, sind die Variablen 
\textit{pg\_pass, authentik\_secret\_key, authentik\_email\_host, authentik\_email\_username, authentik\_host\_password} und \textit{authentik\_email\_from}.
% \textit{PG\_PASS, AUTHENTIK\_SECRET\_KEY, AUTHENTIK\_EMAIL\_\_HOST, AUTHENTIK\_EMAIL\_\_USERNAME, AUTHENTIK\_EMAIL\_\_PASSWORD} 
% und \textit{AUTHENTIK\_EMAIL\_\_FROM}. 
Um Authentik zu starten, werden beide Dateien im Unterordner Authentik eingefügt. 
Der Befehl \textbf{docker-compose pull} lädt die Konfigurationen und Dienste gemäß der Angaben in der \textbf{\nameref{app:docker-compose.yml}} herunter. 
Dabei werden die benötigten Abhängigkeiten und Einstellungen für Container in Docker-Images abgerufen, während Docker-Volumes dafür sorgen, dass 
Daten persistent bleiben, selbst wenn Container gelöscht werden. Anschließend, durch die Ausführung von \textbf{docker-compose up -d}, 
wird sichergestellt, dass die \nameref{app:docker-compose.yml}-Datei vorhanden ist und startet die Container für die definierten Dienste 
im Hintergrund mithilfe des Parameters \textbf{-d (--detach)}. Nach dem Start der Container kann ihr Status mittels \textbf{docker-compose ps} 
überprüft werden. Weitere Einzelheiten und Ergebnisse dieser Befehle findest du im Anhang (\ref{app:dockercommands}) \nameref{app:dockercommands}.

\subsection{Konfiguration des NGinx Reverse Proxy Managers}
\label{sec:Konfiguration des NGinx Reverse Proxy Managers}
Nachdem Authentik in einem Docker-Container läuft, erfolgt die Konfiguration des NGinx Reverse Proxy Managers:
\begin{enumerate}[label=\arabic*.]
    \item Erstellung einer Authentifizierungs-Domain (auth.example.domain) und Einstellung eines A-Records im \acs{DNS}-Resolver
    \item Im NGinx Reverse Proxy Managers einen neuen Host erstellen, was im \nameref{sec:Anhang} in der \ref{app:ProxyHostConfig} \nameref{app:ProxyHostConfig} einsehbar ist
    \item private IP-Adresse des Authentik-Servers mit der in der \textbf{\nameref{app:dotenv}}-Datei eingegebenen Portnummer (80) eingeben 
    mit dem Resultat für das Beispiel: \textbf{0.0.0.0:80} - Einsicht im \ref{app:ProxyHostConfig} \nameref{app:ProxyHostConfig}
    \item im SSL-Tab des Konfigurationsfeldes, anklicken: \textit{Force SSL}, \textit{HTTP/2 Support}, \textit{\acs{HSTS} Enabled} und \textit{\acs{HSTS} Subdomains}
    \item im E-Mail Feld die E-Mail Adresse (admin@example.domain.de) eingeben und den Nutzungsbedingungen zustimmen
    \item Ergebnis: Zugriff auf Authentik möglich, sodass die Willkommens-Seite sichtbar wird, welche im \nameref{sec:Anhang} im Kapitel 
    \ref{app:ProxyHostConfig} \nameref{app:ProxyHostConfig} einsehbar ist
\end{enumerate}

\subsection{Konfiguration von Authentik}
\label{sec: Konfiguration von Authentik}
Nach der Konfiguration des NGinx Reverse Proxy Managers, werden in Authenik ein Projekt und die vorhandenen User erstellt. Diese Vorgehensweise 
ist im \nameref{sec:Anhang} in der \ref{app:AuthentikConfig} \nameref{app:AuthentikConfig} detailliert beschrieben.
\\Grobe Vorgehensweise:
\begin{enumerate}
    \item Einrichtung eines Projektes
    \item Erstellung von Benutzern
    \item Hinzufügen des NGinx Proxy Managers
    \item \nameref{sec:Einrichtung des zweiten Faktors} erfolgt in Kapitel \ref{sec:Einrichtung des zweiten Faktors} auf der 
    Seite \pageref{sec:Einrichtung des zweiten Faktors}
\end{enumerate}


\subsection{Integration mit Cloud-Diensten}
\label{sec:Integration mit Cloud-Diensten}
Damit sich Authentik vor jeden Dienst in der gegebenen Cloud-Infrastruktur schalten kann, wird in der Host-Konfiguration im NGinx Reverse 
Proxy Manager im Bereich \textit{Advanced} die im \nameref{sec:Anhang} erwähnte \nameref{app:CustomNGinxConfig} eingefügt. Wobei zu beachten ist, 
dass bei jedem Dienst die Portnummer geändert werden muss, sodass eine Umleitung von Authentik auf diesen erfolgen kann.

\subsection{Einrichtung des zweiten Faktors}
\label{sec:Einrichtung des zweiten Faktors}
Nach dem Ergebnis aus der \nameref{tab:Nutzwertanalyse} im Kapitel \ref{sec:Auswahl einer MFA-Lösung} \nameref{sec:Auswahl einer MFA-Lösung} 
geht das \acs{TOTP}-Verfahren hervor. Ein Auszug der \nameref{sec:TOTPConfig} im \nameref{sec:Anhang} zeigt die Konfiguration des 
zweiten Faktors.
\\Ein kurze Schrittfolge zur Einrichtung diesens:
\begin{enumerate}
    \item Login bei Authentik
    \item auf \textbf{Einstellungen} gehen und \textbf{\acs{MFA} Devices} bestätigen
    \item \textbf{Enroll} bestätigen - eine Methode auswählen:
    \begin{itemize}
        \item WebAuth Authenticator Setup Stage
        \item Static Authenticator Stage
        \item \textit{TOTP Authenticator Setup Stage} - auswählen
    \end{itemize}
   \item Logout bei Authentik
   \item erneut Login - QR-Code mit dem Microsoft Authenticathor scannen
   \item 6-stelligen Code eingeben und einloggen
\end{enumerate}

\subsection{Maßnahmen zur Qualitätssicherung}
\label{sec:Qualitaetssicherung}

\subsubsection{Produktorientierte Maßnahmen}
\label{sec:ProduktorientierteMaßnahmen}
Die produktorientierten Maßnahmen konzentrieren sich auf die Sicherstellung der Qualität des \acs{MFA}s. Dazu gehören die 
vorzunehmenden Zugriffstests, dargestellt durch Black- and Whitebox-Tests, die die Anforderungen überprüfen und im 
\nameref{sec:Anhang} \ref{app:Testprotokoll} \nameref{app:Testprotokoll} einsehbar sind. Um die Anforderungen zu erfüllen, 
erfolgten die Tests so, dass Birk Spinn und Edgar Johann Kapler ein Feedback auf Korrektheit abgeben werden. 

\subsubsection{Prozessorientierte Maßnahmen}
\label{sec:ProzessorientierteMaßnahmen}
Während der Fokus der prozessorientierten Maßnahmen auf der Qualitätssicherung im Rahmen der Prozesse und Abläufe liegt, die zur 
Implementierung von Authentik gehören. 
Durch die kontinuierliche Anwendung des Qualitätsmanagementsystems gemäß ISO 9001 im Qualitätsmanagement erfolgt nach der Planung 
eine sorgfältige Überprüfung jedes Schrittes auf Richtigkeit. Bei festgestellten Fehlern wird nach alternativen Wegen zur 
Zielerreichung gesucht, um diese erfolgreich umzusetzen. Dieser Prozess folgt nach dem \cite{pdca}-Zyklus.
\\Als Online-Dienstleister ist es laut den Vorschriften des Bundesamts für Sicherheit in der Informationstechnik (BSI) 
zwingend erforderlich, Daten und Geräte durch eine Mehrfaktor-Authentifizierung (\acs{MFA}) zu schützen, bei der der Login durch 
die Verwendung eines Passworts und eines weiteren Faktors abgesichert wird. Authentik, als Identitätsprovider, nutzt ein 
Informationssicherheitssystem, das auf etablierten Standards wie BSI100-2 und ISO27001 basiert und die Anforderungen erfüllt.
Der Standard \acs{BSI} 100-2 ist ein Standard des Bundesamtes für Sicherheit in der Informationstechnik, der allgemeine Anforderungen 
an eine Managementsystem für Informationssicherheit (\acs{ISMS}) definiert. Die ISO 27001 ist eine internationale Norm für \acs{ISMS} 
und ist auf der Basis von IT-Grundschutz für die Standard-Absicherung, als auch für die Kern-Absicherung möglich.