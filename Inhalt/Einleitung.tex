% !TEX root = ../Projektdokumentation.tex
\section{Einleitung}
\label{sec:Einleitung}


\subsection{Projektumfeld} 
\label{sec:Projektumfeld}
Die Deloitte Wirtschaftsprüfungsgesellschaft GmbH ist ein internationales 
Unternehmen für Wirtschaftsprüfung, Steuer-, Unternehmens-, Risiko- und Finanzberatung.
Die Deloitte hat Niederlassungen in vielen Ländern, darunter auch Deutschland. 
Mit Vertreter:innen in über 150 Ländern weltweit und 415.000 Mitarbeiter:innen bietet das Unternehmen eine breite Palette 
von Dienstleistungen für Unternehmen und Organisationen in verschiedenen Branchen.
Im B\&TCL, auch dem Business \& Technology Center Leipzig, erbringt die Deloitte
GmbH mit ihren 100 Mitarbeiter:innen eine Vielfalt an Business Services, mit und ohne IT-Bezug und treibt
Transformationsprojekte rund um die Themen Cyber Security, Digitalisierung,
Prozessoptimierung oder Automatisierung voran.
\\In dem Projekt arbeiten interne Mitarbeiter:innen aktiv mit, um die Entwicklung des
Projektes voranzutreiben. Dabei stellen diese die Zielgruppe dar.


\subsection{Projektziel} 
\label{sec:Projektziel}
Die Deloitte Wirtschaftsprüfungsgesellschaft GmbH verwendet die Cloud-Infrastruktur
von OVHCloud, um einen Business Hosting Service bereitzustellen. 
\\Ziel ist es, die Sicherheit des Zugriffs auf verschiedene Dienste 
innerhalb dieser Cloud-Infrastruktur zu verbessern, indem 
Multi-Faktor-Authentifizierung (MFA) eingeführt wird. 
Dadurch soll eine erhöhte Sicherheit für firmeninterne und kundenbezogene 
Daten gewährleistet werden, indem nur eine Anmeldung mit MFA möglich ist und resultierend Risiken, 
wie Passwork-Leaks und Phishing vermieden werden können.
Um das Ziel zu erreichen wird Authelia für die Implementierung von MFA auf verschiedene Services 
in der Cloud-Infrastruktur eingeführt. 
Dabei ist Authelia eine Identity- und Access-Managment-Lösung (IAM), die den Zugriff auf 
verschiedene Dienste und Ressourcen in der Cloud-Infrastruktur verwalten und sichern soll. 


\subsection{Projektbegründung} 
\label{sec:Projektbegruendung}
Die Einführung der MFA erhöht die Sicherheit und stellt sicher, dass der Zugriff 
auf Dienste und Daten der Cloud-Infrastruktur nicht allein durch ein gestohlenes
Passwort gefährdet sind.
Benutzer müssen zusätzlich zur Eingabe des Passworts einen weiteren Authentifizierungsfaktor, 
wie zum Beispiel ein Einmalpasswort, bereitstellen, was die Sicherheit erheblich erhöht. 
Dabei werden nicht nur firmeninterne und kundenbezogene Daten geschützt, sondern auch 
unsere Kundenzufriedenheit und das Vertrauen erhöht. Dies hat hohe Priorität und 
verhindert unbefugten Zugriff auf sensible Informationen.


\subsection{Projektschnittstellen} 
\label{sec:Projektschnittstellen}

\subsubsection{Technisch}
\label{sec:Technisch}
Die in Kapitel 1.2 besagte Cloud-Infrastruktur wird in vier  
Hauptbereiche unterteilt, sodass in jedem dieser Bereiche verschiedene Services bzw. 
Dienste zur Verfügung werden.
\\\textit{Compliance and Security Stack}
\\Dieser Bereich umfasst den Einsatz von Docker-Containern für Dienste, wie die OPNsense Firewall, 
den Nginx Proxy Manager, Guacamole Server und Infection Monkey zur Sicherheitsüberprüfung. 
Zusammenfassend ist zu sagen, dass der Nginx Proxy Manager als Reverse Proxy fungiert und den HTTP-Verkehr 
umleitet und andere Ports für Streaming-Anforderungen bedient. Das Open-Source-Sicherheitstesttool 
Infection Monkey überprüft die Sicherheit der Infrastruktur. Mit Authelia, als Freeware, wodurch eine wichtige 
Schnittstelle für die Benutzerauthenfizierung und -autorisierung bereitgestellt wird, können von verschiedenen 
Anwendungen über Docker-Container Authentifizierungsmechanismen eingerichtet werden.
\\\textit{Monitoring}
% \begin{center}
% 	\begin{minipage}{0.9\textwidth}
% 	  This is a block of text that is indented from both sides, creating a centered block of text on the page. You can adjust the width by changing the value in the minipage environment.
% 	\end{minipage}
% \end{center}
\\Im Überwachungsbereich werden alle vorhandenen Dienste mittels Docker-Containern der Cloud-Infrastruktur auditiert 
und beinhaltet dieser die Open-Source-Software Grafana zur Visualisierung und Überwachung von Leistungsdaten, 
Uptime Kuma als Webserver für Statusseiten und Healtchchecks, um die Verfügbarkeit der Dienste zu kontrollieren. 
\\\textit{DevOps}
\\Der Fokus dieses Bereiches liegt bei den Anwendungen in der Entwicklung und Bereitstellung dieser und enthält die 
kollaborationsplattform GitLab, um Projekte zu verwalten. Dabei werden diese Dienste mittels Docker-Containern hochgefahren. 
Nexus kommt als Verwaltungstool der Repositories für die Anwendungsabhängigkeiten zum Einsatz. Sonarqube ermöglicht die 
statische Analyse und Bewertung der Quelltextqualität. Zusätzlich wird SFTPGo verwendet, um sichere Authentifizierungsmethoden, 
wie zum Beispiele SSH-Schlüssel und Passwörter zu verwalten.
\\\textit{E-Mail-Kommunikation}
\\In diesem Bereich werden Docker-Container eingesetzt, der den Mail-Server Mailcow verwendet, um E-Mail zu senden und 
zu empfangen, SOGO als Groupware-Lösung und ermöglich somit eine effiziente E-Mail-Kommunikation und Zusammenarbeit innerhalb 
und außerhalb der Organisation.
\subsubsection{Organisatorisch}
\label{sec:Organisatorisch}
Nach der Implementierung Authelias in der Cloud-Umgebung erfolgen erste Tests und Validierungen in den erstellten 
Docker-Containers, um sicherzustellen, dass die MFA ordnungsgemäß funktioniert und den Sicherheitsanforderungen entspricht. 
Im Anschluss erfolgen Schulungen der Benutzer, in dem Fall des Entwicklerteams, zur genauen Verwendung von Authelia mit MFA. 
Nach den Test und Schulungen wird die Authelia-MFA-Implementierung in der Cloud-Umgebung bereitgestellt und kann in die 
kontinuierliche Überwachung und Wartung der Docker-Container sichergestellt werden.

\subsubsection{Personell}
\label{sec:Personell}
Das Projektteam besteht aus folgenden Personen:
\begin{itemize}
	\item Projektauftraggeber/ Director: Herr Dr. Volker Stroetmann
	\item Projektleiter/ Manager/ Projektentwickler: Herr Edgar Johann Kapler
	\item Projektentwickler/ Auszubildender: Herr Dmytro Datsiuk
	\item Projektentwickler/ dualer Student: Herr Neo-Pascal Loest
	\item Projektentwicklerin/ duale Studentin: Frau Martyna Mol 
	\item Projektentwickler/ Auszubildender: Herr Angelo Juliano Vogt
	\item Projektentwicklerin/ Auszubildende: Frau Melissa Futtig
\end{itemize}
Der Projektauftraggeber und -leiter sind die Verantwortlichen für die 
Projektleitung und -finanzierung. Sie genehmigen das Projekt und stellen 
die notwenidgen Ressourcen brereit.
\\Die Projektentwickler: innen sind die allgemeinen Benutzer und das Entwicklerteam, die 
für die Umsetzung und den reibungslosen Betrieb der Cloud-Infrastruktur verantwortlich 
und benötigen sichere Zugriffsmöglichkeiten zu den bereitgestellten Anwendungen.