% !TEX root = ../Projektdokumentation.tex
\section{Einleitung}
\label{sec:Einleitung}


\subsection{Projektumfeld} 
\label{sec:Projektumfeld}
Die Deloitte Wirtschaftsprüfungsgesellschaft GmbHist ein internationales 
Unternehmen für Wirtschaftsprüfung, Steuer-, Unternehmens-, Risiko- und Finanzberatung. 
Die Deloitte hat Niederlassungen in vielen Ländern, darunter auch Deutschland. 
Mit Vertreter:innen in über 150 Ländern weltweit und 415.000 Mitarbeiter:innen bietet das Unternehmen eine breite Palette von Dienstleistungen für Unternehmen und Organisationen in verschiedenen Branchen.
Im B\&TCL, auch dem Business \& Technology Center Leipzig, erbringt die Deloitte
GmbH mit ihren 100 Mitarbeiter:innen eine Vielfalt an Business Services, mit und ohne IT-Bezug und treibt
Transformationsprojekte rund um die Themen Cyber Security, Digitalisierung,
Prozessoptimierung oder Automatisierung voran.
\\In dem Projekt arbeiten interne Mitarbeiter:innen aktiv mit, um die Entwicklung des
Projektes voranzutreiben. Dabei stellen diese die Zielgruppe dar
\begin{itemize}
    \item Kurze Vorstellung des Ausbildungsbetriebs (Geschäftsfeld, Mitarbeiterzahl \usw)
	\item Wer ist Auftraggeber/Kunde des Projekts?
\end{itemize}


\subsection{Projektziel} 
\label{sec:Projektziel}
Die Deloitte Wirtschaftsprüfungsgesellschaft GmbH verwendet die Cloud-Infrastruktur
von OVHcloud, um einen Business Hosting Service bereitzustellen.
\\Ziel ist es, die Sicherheit des Zugriffs auf verschiedene Dienste 
innerhalb dieser Cloud-Infrastruktur zu verbessern, indem 
Multi-Faktor-Authentifizierung (MFA) eingeführt wird. 
Dadurch soll eine erhöhte Sicherheit für firmeninterne und kundenbezogene 
Daten gewährleistet werden.
\begin{itemize}
	\item Worum geht es eigentlich?
	\item Was soll erreicht werden?
\end{itemize}


\subsection{Projektbegründung} 
\label{sec:Projektbegruendung}
Die Einführung der MFA erhöht die Sicherheit und stellt sicher, dass der Zugriff 
auf Dienste und Daten der Cloud-Infrastruktur nicht allein durch ein gestohlenes
Passwort gefährdet ist.
Benutzer müssen zusätzlich zur Eingabe des Passworts einen weiteren Authentifizierungsfaktor, 
wie zum Beispiel ein Einmalpasswort, breitstellen, was die Sicherheit erheblich erhöht. 
Dabei werden nicht nur firmeninterne und kundenbezogene Daten geschützt, sondern auch 
unsere Kundenzufriedenheit und das Vertrauen erhöht. Dies hat hohe Priorität und 
verhindert unbefugten Zugriff auf sensible Informationen. 
\begin{itemize}
	\item Warum ist das Projekt sinnvoll (\zB Kosten- oder Zeitersparnis, weniger Fehler)?
	\item Was ist die Motivation hinter dem Projekt?
\end{itemize}


\subsection{Projektschnittstellen} 
\label{sec:Projektschnittstellen}
\begin{itemize}
	\item Mit welchen anderen Systemen interagiert die Anwendung (technische Schnittstellen)?
	\item Wer genehmigt das Projekt \bzw stellt Mittel zur Verfügung? 
	\item Wer sind die Benutzer der Anwendung?
	\item Wem muss das Ergebnis präsentiert werden?
\end{itemize}


\subsection{Projektabgrenzung} 
\label{sec:Projektabgrenzung}
\begin{itemize}
	\item Was ist explizit nicht Teil des Projekts (\insb bei Teilprojekten)?
\end{itemize}
