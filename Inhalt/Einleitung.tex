% !TEX root = ../Projektdokumentation.tex
\section{Einleitung}
\label{sec:Einleitung}


\subsection{Projektumfeld} 
\label{sec:Projektumfeld}
Die Deloitte Wirtschaftsprüfungsgesellschaft GmbHist ein internationales 
Unternehmen für Wirtschaftsprüfung, Steuer-, Unternehmens-, Risiko- und Finanzberatung. 
Die Deloitte hat Niederlassungen in vielen Ländern, darunter auch Deutschland. 
Mit Vertreter:innen in über 150 Ländern weltweit und 415.000 Mitarbeiter:innen bietet das Unternehmen eine breite Palette von Dienstleistungen für Unternehmen und Organisationen in verschiedenen Branchen.
Im B\&TCL, auch dem Business \& Technology Center Leipzig, erbringt die Deloitte
GmbH mit ihren 100 Mitarbeiter:innen eine Vielfalt an Business Services, mit und ohne IT-Bezug und treibt
Transformationsprojekte rund um die Themen Cyber Security, Digitalisierung,
Prozessoptimierung oder Automatisierung voran.
\\In dem Projekt arbeiten interne Mitarbeiter:innen aktiv mit, um die Entwicklung des
Projektes voranzutreiben. Dabei stellen diese die Zielgruppe dar.


\subsection{Projektziel} 
\label{sec:Projektziel}
Die Deloitte Wirtschaftsprüfungsgesellschaft GmbH verwendet die Cloud-Infrastruktur
von OVHCloud, um einen Business Hosting Service bereitzustellen. 
\\Ziel ist es, die Sicherheit des Zugriffs auf verschiedene Dienste 
innerhalb dieser Cloud-Infrastruktur zu verbessern, indem 
Multi-Faktor-Authentifizierung (MFA) eingeführt wird. 
Dadurch soll eine erhöhte Sicherheit für firmeninterne und kundenbezogene 
Daten gewährleistet werden, indem nur eine Anmeldung mit MFA möglich ist dadurch Risiken, 
wie Passwork-Leaks und Phishing vermieden werden können.


\subsection{Projektbegründung} 
\label{sec:Projektbegruendung}
Die Einführung der MFA erhöht die Sicherheit und stellt sicher, dass der Zugriff 
auf Dienste und Daten der Cloud-Infrastruktur nicht allein durch ein gestohlenes
Passwort gefährdet ist.
Benutzer müssen zusätzlich zur Eingabe des Passworts einen weiteren Authentifizierungsfaktor, 
wie zum Beispiel ein Einmalpasswort, breitstellen, was die Sicherheit erheblich erhöht. 
Dabei werden nicht nur firmeninterne und kundenbezogene Daten geschützt, sondern auch 
unsere Kundenzufriedenheit und das Vertrauen erhöht. Dies hat hohe Priorität und 
verhindert unbefugten Zugriff auf sensible Informationen.


\subsection{Projektschnittstellen} 
\label{sec:Projektschnittstellen}
Die in Kapitel 1.2 besagte Cloud-Infrastruktur wird in vier Hauptbereiche unterteilt, sodass in jedem 
dieser Bereiche verschiedene Services bzw. Dienste bereitgestellt werden. 
\\Beginnend mit dem 'Compliance and Security Stack', dem wörtlich übersetzten Ordnungsmäßigkeits- und 
Sicherheitsbereich, welcher eine OPNsense Firewall umfasst, einen Nginx Proxy Manager, der als 
Reverse Proxy den HTTP-Verkehr umleitet und andere Ports für spezielle Streaming-Anforderungen bedient. 
Desweitren kommt der Guacamole Server zum Einsatz, um sichere Fernzugriffe zu ermöglichen, sowie das 
Open-Source-Sicherheitstesttool Infection Monkey, um die Sicherheit der Infrastruktur zu überprüfen. 
Der Rahmen 'Monitoring', dem Überwachungsbereich, werden alle vorhandenen Dienste der Cloud-Infrastruktur 
auditiert und beinhaltet die Open-Source-Software Grafana zur Visualisierung und Überwachung von 
Leistungsdaren, Uptime Kuma als Webserver für Statusseiten und Healthchecks, um die Verfügbarkeit 
der Dienste zu kontrollieren. 
Im dritten und 'DevOps'-Bereich, werden die Entwicklung und Bereitstellung von Anwendungen in den 
Fokus gestellt und enthält dieser die kollaborative Entwicklugsplattform GitLab, um Projekte zu 
verwalten. Nexus kommt zur Verwaltung von Repositories für die Anwendungsabhängigkeiten zum Einsatz. 
Sonarqube ermöglicht die statische Analyse und Bewertung der Quelltextqualität. Zusätzlich wird SFTPGo 
verwendet, um sichere Authentifizierungmethoden, wie zum Beispiel öffentliche Schlüssel, SSH-Schlüssel 
und Passwörter zu verwalten. 
Beendend mit dem 'E-Mail'-Bereich, in welchem die die E-Mail-Kommunikation stattfindent, wird der 
Mail-Server Mailcow verwendet, um E-Mails zu senden und zu empfangen, SOGO als Groupware-ösung und 
ermöglicht somit eine effiziente E-mail-Kommunikation und Zusammenarbeit innerhalb und außerhalb der Organisation. 
Diese vier Bereiche sind notwendig, um den reibungslosen Betrieb der Cloud-Infrastruktur zu gewährleisten.
\\Die Projektgenehmigung erfolgt durch den Director und Projektauftraggeber Volker Stroetmann und dessen 
Projektmanager Edgar Kapler, welche dem Projekt die nötigen Mittel zur Verfügung stellen.
\\Die allgemeinen Benutzer sind interne und externe Kunden, welche einen sicheren Zugriff auf ihre 
zu nutzenden Anwendungen haben. ??? Nochmal in Präsentation JourFixe nachlesen!! ???
%Authelia selbst ist eine wichtige Schnittstelle für die Benutzerauthentifizierung und 
%-autorisierung bietet eine API, die von verschiedenen Anwendungen genutzt werden kann, 
%um den benötigten Authentifizierungsmechanismus einzurichten.
