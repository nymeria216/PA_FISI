
% !TEX root = ../Projektdokumentation.tex
\section{Einleitung}
\label{sec:Einleitung}


\subsection{Projektumfeld} 
\label{sec:Projektumfeld}
Die Deloitte Wirtschaftsprüfungsgesellschaft GmbH ist ein internationales 
Unternehmen für Wirtschaftsprüfung, Steuer-, Unternehmens-, Risiko- und Finanzberatung.
Die Deloitte hat Niederlassungen in vielen Ländern, darunter auch Deutschland. 
Mit Vertreter: innen in über 150 Ländern weltweit und 415.000 Mitarbeiter: innen bietet das Unternehmen eine breite Palette 
von Dienstleistungen für Unternehmen und Organisationen in verschiedenen Branchen.
Im B\&TCL, auch dem Business \& Technology Center Leipzig, erbringt die Deloitte
GmbH mit ihren 100 Mitarbeiter: innen eine Vielfalt an Business Services, mit und ohne IT-Bezug und treibt
Transformationsprojekte rund um die Themen Cyber Security, Digitalisierung,
Prozessoptimierung oder Automatisierung voran.
\\Das Projekt ist ein Tochterprojekt eines größeren und beinhaltet die Implementierung eines Teilfeatures, was zur 
Verbesserung des Gesamtprojektes führt. In diesem arbeiten interne Mitarbeiter: innen aktiv mit, um die Entwicklung des
Projektes voranzutreiben. Dabei stellen diese die Zielgruppe dar.


\subsection{Projektziel} 
\label{sec:Projektziel}
Die Deloitte Wirtschaftsprüfungsgesellschaft GmbH verwendet die Cloud-Infrastrukturvon OVHCloud, um einen Business 
Hosting Service bereitzustellen. 
\\Ziel ist es, die Sicherheit des Zugriffs auf verschiedene Dienste innerhalb dieser Cloud-Infrastruktur zu verbessern, indem 
Multi-Faktor-Authentifizierung (MFA) eingeführt wird. Dadurch soll eine erhöhte Sicherheit für firmeninterne und kundenbezogene 
Daten gewährleistet werden, indem nur eine Anmeldung mit MFA möglich ist und resultierend daraus Risiken, wie Passwork-Leaks 
und Phishing vermieden werden können.
Um das Ziel zu erreichen wird Authentik für die Implementierung von MFA auf verschiedene Services in der 
Cloud-Infrastruktur eingeführt. 
Dabei ist Authentik ein Open-Source-Identitätsanbieter (Identity Provider), der den Zugriff auf verschiedene Dienste und 
Ressourcen in der Cloud-Infrastruktur verwalten und sichern soll und legt den Fokus auf die Flexibilität und Vielseitigkeit. 


\subsection{Projektbegründung} 
\label{sec:Projektbegruendung}
Die Implementierung der MFA steigert die Sicherheit und gewährleistet, dass der Zugang zu Diensten und Daten in der 
Cloud-Infrastruktur nicht allein durch den Diebstahl eines Passworts gefährdet ist. Benutzer müssen zusätzlich zur Eingabe des 
Nutzernamens und Passworts einen weiteren Authentifizierungsfaktor, wie zum Beispiel ein Einmalpasswort, bereitstellen, 
was die Sicherheit erheblich erhöht. Dabei werden nicht nur firmeninterne und kundenbezogene Daten geschützt, sondern auch 
die Kundenzufriedenheit und das Vertrauen gegenüber der zukünftigen Kunden erhöht. Dies hat hohe Priorität und verhindert 
unbefugten Zugriff auf sensible Informationen.


\subsection{Projektschnittstellen} 
\label{sec:Projektschnittstellen}

\subsubsection{Technisch}
\label{sec:Technisch}
Die in Kapitel 1.2 besagte Cloud-Infrastruktur wird in vier  
Hauptbereiche unterteilt, sodass in jedem dieser Bereiche verschiedene Services bzw. 
Dienste zur Verfügung werden.
\\\textit{Compliance and Security Stack}
\\Dieser Bereich umfasst den Einsatz von Docker-Containern für Dienste, wie die OPNsense Firewall, 
den Nginx Proxy Manager, dem Authentik- und Guacamole Server und Infection Monkey zur Sicherheitsüberprüfung. 
Zusammenfassend ist zu sagen, dass der Nginx Proxy Manager als Reverse Proxy fungiert und den HTTP-Verkehr 
umleitet und andere Ports für Streaming-Anforderungen bedient. Der Guacamole Server dient als Proxy-Server 
und ermöglicht den Zugriff auf interne Dienste, die normalerweise nicht über eine Web-Schnittstelle erreichbar sind.
Das Open-Source-Sicherheitstesttool Infection Monkey überprüft die Sicherheit der Infrastruktur. 
Mit dem Identity Provider Authentik, werden verschiedene Identitäts- und Authentifizierungsmethoden in 
Anwendungen und Diensten ermöglicht zu integrieren. Dadurch wird eine wichtige Schnittstelle für die 
Benutzerauthenfizierung und -autorisierung bereitgestellt und kann von verschiedenen Anwendungen über 
Docker-Container Authentifizierungsmechanismen einrichten.
\\\textit{Monitoring}
% \begin{center}
% 	\begin{minipage}{0.9\textwidth}
% 	  This is a block of text that is indented from both sides, creating a centered block of text on the page. You can adjust the width by changing the value in the minipage environment.
% 	\end{minipage}
% \end{center}
\\Im Überwachungsbereich werden alle vorhandenen Dienste mittels Docker-Containern der Cloud-Infrastruktur auditiert 
und beinhaltet dieser die Open-Source-Software Grafana zur Visualisierung und Überwachung von Leistungsdaten, sowie 
Uptime Kuma als Webserver für Statusseiten und Healtchchecks, um die Verfügbarkeit der Dienste zu kontrollieren. 
\\\textit{DevOps}
\\Der Fokus diesen Bereiches liegt bei den Anwendungen in der Entwicklung und Bereitstellung und enthält die 
Kollaborationsplattform GitLab, um Projekte zu verwalten. Dabei werden diese Dienste mittels Docker-Containern hochgefahren. 
Nexus kommt als Verwaltungstool der Repositories für die Anwendungsabhängigkeiten zum Einsatz. Sonarqube ermöglicht die 
statische Analyse und Bewertung der Quelltextqualität. Zusätzlich wird SFTPGo verwendet, um sichere Authentifizierungsmethoden, 
wie zum Beispiele SSH-Schlüssel und Passwörter zu verwalten.
\\\textit{E-Mail-Kommunikation}
\\In diesem Bereich werden Docker-Container eingesetzt, der den Mail-Server Mailcow hochfährt, um E-Mails zu senden und 
zu empfangen, SOGO als Groupware-Lösung und ermöglich dadurch eine effiziente E-Mail-Kommunikation und Zusammenarbeit innerhalb 
und außerhalb der Organisation.
Insgesamt werden für die Infrastruktur und dieses Projekt drei Instanzen verwendet. Zum Einen ist es die \textbf{''control\_node''} 
als eine Instanz, auf welcher sich zu Beginn die Benutzer :innen verbinden, woraufhin die folgende Verbindung auf die nächste Instanz, 
entweder der auf den \textbf{''rev\_prox\_dev''} oder der \textbf{''tal\_cloud\_infra''} über SSH passiert. Dabei befinden 
sich alle Services in den oben genannten Bereichen auf der \textbf{''tal\_cloud\_infra''} außer 
der Nginx Proxy, der auf dem \textbf{''rev\_prox\_dev''} lokalisiert ist. Eine Vorschau der Cloud-Infrastruktur befindet 
sich im \nameref{sec:Anhang} im Kapitel \ref{app:Cloud-Infrastruktur} in der \nameref{app:Cloud-Infrastruktur} auf der Seite 
\pageref{app:Cloud-Infrastruktur}.

\subsubsection{Organisatorisch}
\label{sec:Organisatorisch}
Nach der Implementierung Authentiks in der Cloud-Umgebung, erfolgen erste Tests und Validierungen in den erstellten 
Docker-Containern via 'docker-compose.yml-Dateien', um sicherzustellen, dass die MFA ordnungsgemäß funktioniert und den 
Sicherheitsanforderungen entspricht. 
Im Anschluss erfolgen Schulungen der Benutzer, in dem Fall das Entwicklerteam, zur genauen Verwendung von Authentik mit MFA. 
Nach den Tests und der Schulungen wird die Authentik-MFA-Implementierung in der Cloud-Umgebung bereitgestellt.

\subsubsection{Personell}
\label{sec:Personell}
Das Projektteam besteht aus folgenden Personen:
\begin{itemize}
	\item Projektauftraggeber/ Director: Herr Dr. Volker Stroetmann
	\item Projektleiter/ Manager/ Projektentwickler: Herr Edgar Johann Kapler
	\item Projektentwickler/ Auszubildender: Herr Dmytro Datsiuk
	\item Projektentwickler/ dualer Student: Herr Neo-Pascal Loest
	\item Projektentwicklerin/ duale Studentin: Frau Martyna Mol 
	\item Projektentwickler/ Auszubildender: Herr Angelo Juliano Vogt
	\item Projektentwicklerin/ Auszubildende: Frau Melissa Futtig
\end{itemize}
Der Projektauftraggeber und -leiter sind die Verantwortlichen für die Projektleitung und -finanzierung des Projektes. 
Sie genehmigen dieses und stellen die notwendigen Ressourcen zur Entwicklung bereit.
\\Die Projektentwickler: innen sind die allgemeinen Benutzer und das Entwicklerteam. Sie sind für die Umsetzung und den 
reibungslosen Betrieb der Cloud-Infrastruktur verantwortlich und benötigen sichere Zugriffsmöglichkeiten zu den 
bereitgestellten Anwendungen.