
% !TEX root = ../Projektdokumentation.tex
\section{Einleitung}
\label{sec:Einleitung}

\subsection{Projektumfeld} 
\label{sec:Projektumfeld}
Die \cite{Deloitte} Wirtschaftsprüfungsgesellschaft GmbH ist ein internationales 
Unternehmen für Wirtschaftsprüfung, Steuer-, Unternehmens-, Risiko- und Finanzberatung.
Sie hat Niederlassungen in vielen Ländern, darunter auch Deutschland. 
Mit Vertretern in über 150 Ländern weltweit und 415.000 Mitarbeitern, bietet das Unternehmen eine breite Palette 
von Dienstleistungen für Unternehmen und Organisationen in verschiedenen Branchen.
Im \acs{B and TCL}, auch dem Business \& Technology Center Leipzig, erbringt die Deloitte
GmbH mit ihren 100 Mitarbeitern eine Vielfalt an Business Services, mit und ohne IT-Bezug und treibt
Transformationsprojekte rund um die Themen Cyber Security, Digitalisierung,
Prozessoptimierung oder Automatisierung voran. 
\\Das Projekt mit dem Fokus auf Multi-Faktor-Authentifizierung bei verschiedenen Services in einer 
Cloud-Infrastruktur ist ein Tochterprojekt eines Größeren. Dabei ist gemeint, dass dieses 
''Tochterprojekt'' nicht alleine existiert, sondern als Teil des größeren Projektes agiert und es sich 
um ein Subprojekt handelt. Die Deloitte Wirtschaftsprüfungsgesellschaft GmbH verwendet die 
Cloud-Infrastruktur von \cite{ovhcloud}, um einen Business Hosting Service bereitzustellen.
Die Zielgruppe stellt das Projektteam dar, welche in dem größeren Projekt mit entwickeln.

\subsection{Projektaufgabe}
\label{sec:Projektaufgabe}
Der Zweck des Projektes ist es, den Entwicklern der Cloud-Infrastruktur mit den darauf gehosteten Services bzw. 
Diensten, eine Authentifizierungsmöglichkeit zu bieten, die die Sicherheit des Einloggens erhöht.
Die Implementierung der \acs{MFA} steigert nicht nur die Sicherheit, sondern gewährleistet gleichermaßen, 
dass der Zugang zu den Diensten und Daten in der Cloud-Infrastruktur nicht allein durch den Diebstahl eines 
Passworts gefährdet wird. Benutzer müssen zusätzlich zur Eingabe des Nutzernamens und Passwortes einen weiteren 
Authentifizierungsfaktor, wie zum Beispiel ein Einmalkennwort, eingeben. Dabei werden nicht nur firmeninterne und 
kundenbezogene Daten geschützt, sondern auch die Kundenzufriedenheit und das Vertrauen gegenüber der zukünftigen 
Kunden erhöht. Dies hat hohe Priorität und verhindert unbefugten Zugriff auf sensible Informationen.

\subsection{Projektziel} 
\label{sec:Projektziel}
Ziel ist es, die Sicherheit des Zugriffs auf verschiedene Dienste innerhalb dieser Cloud-Infrastruktur zu verbessern, indem 
Multi-Faktor-Authentifizierung (\acs{MFA}) eingeführt wird. Dadurch soll eine erhöhte Sicherheit für firmeninterne und kundenbezogene 
Daten gewährleistet werden, indem nur eine Anmeldung mit \acs{MFA} möglich ist und resultierend daraus Risiken, wie Passwork-Leaks 
und Phishing vermieden werden können.
Um das Ziel zu erreichen, soll eine MFA-Lösung implementiert werden, welche sich vor die verschiedenen Dienste in der 
Cloud-Infrastruktur schaltet und den Zugriff der Benutzer verwalten und sichern soll.

\subsection{Projektschnittstellen} 
\label{sec:Projektschnittstellen}

\subsubsection{Technisch}
\label{sec:Technisch}
Die in Kapitel 1.2 besagte Cloud-Infrastruktur wird in vier  
Hauptbereiche unterteilt, sodass in jedem dieser Bereiche verschiedene Dienste zur Verfügung werden. In jedem dieser Bereiche 
werden die Dienste mittels Docker-Containern initialisiert.
\\\textit{Compliance and Security Stack}
\\Dieser Bereich umfasst den Einsatz, wie die OPNsense Firewall, den NGinx Proxy Manager, dem Authentik- und Guacamole Server 
und Infection Monkey zur Sicherheitsüberprüfung. Zusammenfassend ist zu sagen, dass der NGinx Proxy Manager als Reverse Proxy 
fungiert und den HTTP-Verkehr umleitet und andere Ports für Streaming-Anforderungen bedient. Mittels von Apache Guacamole 
erfolgt die Sicherstellung des Fernzugriffs auf die internen Dienste, die regulär nicht über eine Web-Schnittstelle erreichbar sind.
Das Open-Source-Sicherheitstesttool Infection Monkey überprüft die Sicherheit der Infrastruktur. 
Mit dem Identitätsanbieter \cite{Authentik}, werden verschiedene Identitäts- und Authentifizierungsmethoden in 
Anwendungen und Diensten ermöglicht, zu integrieren. Dadurch wird eine wichtige Schnittstelle für die 
Benutzerauthenfizierung und -autorisierung bereitgestellt und kann von verschiedenen Anwendungen über 
Docker-Container Authentifizierungsmechanismen einrichten.
\\\textit{Monitoring}
\\Im Überwachungsbereich agiert die Open-Source-Software Grafana für die Visualisierung und Überwachung von Leistungsdaten, sowie 
Uptime Kuma als Webserver für Statusseiten und Healtchchecks, um die Verfügbarkeit der Dienste zu kontrollieren. 
\\\textit{DevOps}
\\Der Fokus liegt bei den Anwendungen in der Entwicklung und Bereitstellung und enthält die 
Kollaborationsplattform GitLab, um beispielsweise Projekte zu verwalten. 
Nexus kommt als Verwaltungstool der Repositories für die Anwendungsabhängigkeiten zum Einsatz. Sonarqube ermöglicht die 
statische Analyse und Bewertung der Quelltextqualität. Zusätzlich wird SFTPGo verwendet, um sichere Authentifizierungsmethoden, 
wie zum Beispiele SSH-Schlüssel und Passwörter zu verwalten.
\\\textit{E-Mail-Kommunikation}
\\In diesem Bereich agiert der Mail-Server Mailcow, um E-Mails zu senden und empfangen, SOGO als Groupware-Lösung, 
was eine effiziente E-Mail-Kommunikation und Zusammenarbeit innerhalb und außerhalb des Unternehmens ermöglicht.
\\Für dieses Projekt werden insgesamt drei Instanzen des Anbieters OVHCloud für Cloud-Computing-Dienste genutzt. 
Alle drei Instanzen sind Linux-Systeme mit Ubuntu und variieren in ihren Ressourcen wie RAM und CPU. Auf 
diesen Instanzen werden die im Kapitel genannten Dienste durch Containerisierung gestartet. Jeder Instanz 
wird zur besseren Zuordnung eine spezifische Bezeichnung zugeordnet.
Die erste Instanz, genannt \textbf{''control\_node''}, fungiert als Anlaufstelle, an der sich Benutzer zu Beginn verbinden müssen.
Diese Instanz dient als Kontrollpunkt, über den der Zugang zu den anderen Instanzen erfolgt. Sie fungiert 
gewissermaßen als Gateway zu entweder der Instanz \textbf{''rev\_prox\_dev''} oder der \textbf{''tal\_cloud\_infra''}. 
Der Zugriff auf diese Instanzen geschieht mittels eines \acs{SSH}-Schlüssels von einem lokalen Notebook aus über die Kommandozeile.
Alle Dienste, bis auf den NGinx Reverse Proxy Manager, sind in der ''tal\_cloud\_infra'' lokalisiert. 
Letzterer läuft auf der ''rev\_prox\_dev''-Instanz. Die Bezeichnung ‘‘tal\_cloud\_infra‘‘ bezieht sich auf den Standort Deloitte in Leipzig, bekannt als \acs{TAL}. 
''rev\_prox\_dev'' erhält seinen Namen aufgrund des dort laufenden NGinx Reverse Proxy Managers. Ein Diagramm, das den Zugriff eines Benutzers auf eine Instanz  
über die CLI veranschaulicht, befindet sich im \nameref{sec:Anhang} \nameref{app:Sequenzdiagramm CLI Zugriff auf die Instanzen}.


\subsubsection{Organisatorisch}
\label{sec:Organisatorisch}
Nach der Implementierung der MFA-Lösung in der Cloud-Umgebung, finden erste Tests und Validierungen in den erstellten 
Docker-Containern via der \textit{docker-compose.yml}-Datei statt, um sicherzustellen, dass diese ordnungsgemäß funktioniert und den 
Sicherheitsanforderungen entspricht. 
Im Anschluss erfolgen Schulungen der Benutzer, in dem Fall für das Entwicklerteam, zur genauen Verwendung von Authentik mit \acs{MFA}. 
Nach den Tests und der Schulungen wird die Authentik-\acs{MFA}-Implementierung in der Cloud-Umgebung bereitgestellt.

\subsubsection{Personell}
\label{sec:Personell}
Das Projektteam besteht aus folgenden Personen:
\begin{itemize} [label=--]
	\item Projektleiter/ Manager/ Projektentwickler: Herr Edgar Johann Kapler
	\item Projektentwickler/ dualer Student: Herr Birk Spinn
	\item Projektentwicklerin/ Auszubildende: Frau Melissa Futtig
\end{itemize}
Der Projektleiter ist der Verantwortliche für die Projektleitung und -finanzierung des Projektes. 
Er genehmigt dieses und stellt die notwendigen Ressourcen zur Entwicklung zur Verfügung.
\\Die Projektentwickler sind die allgemeinen Benutzer \bzw das Entwicklerteam. Sie sind für die Umsetzung und den 
reibungslosen Betrieb der Cloud-Infrastruktur verantwortlich und benötigen sichere Zugriffsmöglichkeiten zu den 
bereitgestellten Anwendungen.