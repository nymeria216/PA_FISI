\section{Einleitung}
\label{sec:Einleitung}

Die folgende Arbeit beschäftigt sich mit der Verwendung von Rhetorik in 
politischen Reden und ihre Auswirkungen auf das Publikum. \\Ziel dieser Facharbeit 
ist es, anhand politischer Reden die folgende Fragestellung \textit{“In wie weit 
beeinflussen rhetorische Mittel in politischen Reden das Verhalten des 
Publikums?“} zu beantworten. \\Die Erforschung dieses Themas ist wichtig, um ein 
besseres Verständnis dafür zu entwickeln, wie politische Kommunikation 
funktioniert und wie sie die Gesellschaft beeinflusst. \\Ich habe mich für dieses 
Thema entschieden, weil Psychologie und Manipulation spannende und interessante Themen sind, 
bei denen ich meine Fähigkeiten im privaten Umfeld weiter ausbauen kann. 
Desweiteren erhalte ich einen großen Einblick in die Art und Funktionsweise von 
Rhetorik, um eine Manipulation meinerseits vermeiden zu können. 
Der praktische Teil besteht aus zwei politischen Reden, einem Vergleich beider 
und dem Aufzeigen über den Einfluss von rhetorischen Mitteln auf das Publikum \bzw die Bevölkerung.
Zum einen handelt es sich um die \nameref{sec: Antrittsrede von Donald Trump} am
20.01.2017 und zum nderen um die \nameref{sec: Coronarede von Angela Merkel} am 18.03.2020. 
Im Anschluss erfolgt im zweiten Kapitel \nameref{sec:Rhetorik} eine systematische 
Schilderung des Begriffes selbst und des historischen Hintergunds mit den 
jeweiligen vorhandenen Säulen und der Erklärung von rhetorischen Mitteln. 
\\Das dritte Kapitel \nameref{sec: Analyse der Reden} ist die praktische 
Ausarbeitung. Das vierte Kapitel \nameref{sec: Fazit} 
behinhaltet eine Zusammenfassung und Standpunktnahme mit einer Reflexion der Arbeit.
\cleardoublepage