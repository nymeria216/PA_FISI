\section{Problemstellung}
\label{sec:Problemstellung}
Die zunehmende Bedeutung von Cloud-Infrastrukturen für Unternehmen geht einher mit der Herausforderung, 
eine robuste Sicherheitsarchitektur zu etablieren. Die Deloitte Wirtschaftsprüfungsgesellschaft GmbH, 
als Anbieter von umfassenden Unternehmensdienstleistungen, steht vor der Notwendigkeit, die Sicherheit 
und Vertraulichkeit ihrer Cloud-basierten Dienste zu erhöhen. Aktuelle Sicherheitsmechanismen reichen 
nicht aus, um den ständig wachsenden Bedrohungen und Angriffen auf Unternehmensdaten wirksam zu 
begegnen. Insbesondere der Zugang zu sensiblen Informationen über die Cloud-Infrastruktur birgt ein 
hohes Risiko, das durch die herkömmliche Authentifizierung mittels Benutzername und Passwort allein 
nicht ausreichend adressiert wird.
\\Das Fehlen einer zusätzlichen Sicherheitsebene in Form einer Multi-Faktor-Authentifizierung (MFA) 
lässt potenzielle Schwachstellen offen, die es Angreifern ermöglichen könnten, über gestohlene 
Zugangsdaten unbefugten Zugriff zu erlangen. Diese Lücke in der Sicherheitsstrategie unterstreicht die 
Notwendigkeit eines Projektes, das die Einführung einer MFA-Lösung für die Cloud-Infrastruktur der 
Deloitte GmbH fokussiert. Durch die Implementierung einer MFA wird nicht nur die Sicherheit des 
Zugriffs auf die Dienste erhöht, sondern auch das Vertrauen der Kunden gestärkt und das Risiko von 
Datenlecks und Phishing-Angriffen minimiert.