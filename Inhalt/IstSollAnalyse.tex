% !TEX root = ../Projektdokumentation.tex
\section{Ist/Soll-Analyse} 
\label{sec:IstSollAnalyse}


\subsection{Ist-Analyse} 
\label{sec:IstAnalyse}
Die aktuelle Cloud-Infrastruktur nutzt den NGinx Reverse Proxy Manager für den Zugriff auf verschiedene Dienste. 
Es besteht jedoch keine zusätzliche Sicherheitsschicht wie eine Multi-Faktor-Authentifizierung (\acs{MFA}), 
was bedeutet, dass der Zugriff allein über Benutzernamen- und Passwörtereingabe erfolgt, ohne eine weitere 
Überprüfung der Identität.
% Die Cloud-Umgebung enthält die in den oberen Kapiteln erwähnten Instanzen 
% (\textit{control\_node, rev\_prox\_dev und tal\_cloud\_infra}) und eine OPNSense-Firewall, die den Zugriff 
% von unbefugten Dritten verhindern soll. Die drei Instanzen befinden sich in einem privaten Netzwerk, was 
% bedeutet, dass diesen private nicht von außen sichtbare IP-Adressen zugewiesen werden, während der 
% Firewall eine öffentliche IP-Adresse vergeben wird. 
% Um einen Zugriff auf die Instanzen zu ermöglichen, wird den Command-Line-Interface-Nutzern (\acs{CLI}) mit einem gültigen
% SSH-Schlüssel die Möglichkeit geboten, von der Kontrollinstanz \textit{control\_node} auf die anderen Instanzen 
% zuzugreifen. Die anderen für das Projekt relevanten Instanzen sind \textit{rev\_proxy\_dev} und \textit{tal\_cloud\_infra} 
% auf denen sich, die im Kapitel \ref*{sec:Technisch} \nameref{sec:Technisch}e \nameref{sec:Projektschnittstellen} 
% erwähnten Instanzen befinden. Die Graphical-User-Interface-Nutzer (\acs{GUI}) können die Services über den NGinx Reverse Proxy Manager 
% zu erreichen, indem der Zugriff klassisch mittels einer einfachen Nutzer- und Passworteingabe, ohne einer weiteren 
% Schutzebene erfolgt.

\subsection{Soll-Analyse}
\label{sec:SollAnalyse}
Das Hauptziel besteht darin, die Sicherheit der Dienste durch die Implementierung einer \acs{MFA}-Lösung 
zu verbessern. Geplant ist, vor jedem Dienst einen Identitätsanbieter zu schalten, um sicherzustellen, 
dass sich die \acs{MFA} vor den Diensten positioniert. Benutzer sollen sich über die \acs{MFA} 
authentifizieren, um den Identitätsbestätigungsprozess zu erfüllen und Zugang zu den Diensten zu 
erhalten.


\subsection{Wirtschaftlichkeitsanalyse}
\label{sec:Wirtschaftlichkeitsanalyse}
Durch die schon vorhandene Cloud-Infrastruktur des größeren Projektes entstehen keine weiteren Kosten. Bedingt dessen, dass die 
OVH-Cloud-Infrastruktur zu einem Fix-Preis pro Instanz gemietet wird und keine weiteren Instanzen für die Implementierung 
des Tochter-Projektes erforderlich sind, bleiben die Kosten unverändert. 
An ressourcen wird zwar mehr CPU und Rechenleistung verwendet, was allerdings nicht nach dem \cite{Wiki}-Prinzip 
berechnet wird, sondern bei OVHCloud pauschal nach Instanzpreis, sodass für das ''Tochter-Projekt'' keine weitere 
Kosten entstehen. Die wirklich zu entstehenden Kosten sind ausschließlich Personal- und Materialkosten, wobei letztere 
aus der Nutzung des Büromobiliars und der Strom- und Heizkosten zusammengefasst wird. 
\\Durch die Einführung der \acs{MFA}-Lösung in der Cloud-Infrastruktur werden besonders die Schutzziele der 
Einhaltung der Integrität und Vertraulichkeit der Daten auf den jeweiligen Diensten eingehalten. So wird die Sicherheit 
erheblich verbessert und trägt dazu bei, unbefugten Zugriff, Datenverluste und Betrug zu verhindern. Dieser Schutz 
vor Sicherheitsverletzungen kann erhebliche finanzielle Auswirkungen haben, da die Wiederherstellungskosten vermieden 
werden können. Des Weiteren erfolgen Kostenersparnisse in dem Punkt des Verhinderns der Passwort-Resets durch das Anfragen des 
administrativen Supports, bei welchem Zeit und daraus Kosten entstehen, die vermieden werden können. Durch die Implementierung 
kann den Nutzern ein sicherer und bequemerer Zugriff gewährleistet und die Kosten gesenkt werden.


\subsection{Anforderungsanalyse}
\label{sec:Anforderungsanalyse}

\subsubsection{Funktionale Anforderungen}
\label{sec:Funktional}
Die MFA-Lösung muss folgende funktionale Anforderungen erfüllen:
\begin{itemize} [label=--]
	\item Benutzerfreundliche Registrierung und Login
	\item Klare Anleitung für die Einmalpasswort-Eingabe
	\item Übersichtliche Verwaltung von Nutzern
	\item Einloggen als Admin und Benutzer
	\item Dienste nach Benutzerdateneingabe aktivieren
	\item Schutz vor Drittzugriff, sichere Lösung
	\item Nutzung mit Authentikator-Software
\end{itemize}

\subsubsection{Nicht-Funktionale Anforderungen}
\label{sec:Nicht-Funktional}
Authentik muss folgende nicht-funktionale Anforderungen erfüllen:
\begin{itemize} [label=--]
	\item Skalierbarkeit der Dienste
	\item Übergang zu Diensten in bis zu 2 Sekunden
	\item Übersichtliches Willkommensdisplay (Englisch)
\end{itemize}

% \subsubsection{\gqq{Make or Buy}-Entscheidung}
% \label{sec:MakeOrBuyEntscheidung}
% In der Entscheidungsmatrix für die Zielplattform aus dem Kapitel~\ref{sec:Authentifizierungs-Tool}: \nameref{sec:Authentifizierungs-Tool}
% auf der Seite \pageref{sec:Authentifizierungs-Tool} sind einige alternative Produkte sichtbar, welche wie Authentik, 
% implementiert werden müssen. Daraus resultierend wird sich für Authentik entschieden.

\subsection{Anwendungsfälle}
\label{sec:Anwendungsfaelle}
Ein Use Case-Diagramm zur Veranschaulichung des Prozesses der Cloud-Infrastruktur findet sich im \nameref{sec:Anhang} \ref{app:UseCase} \nameref{app:UseCase}.
In diesem interagiert der Akteur aus der Sicht eines Projektentwicklers mit dem System, in welchem verschiedene 
Anwendungsfälle existieren. Der Akteur meldet sich über den Authentik Server bei der Firewall und dem Nginx Proxy Manager an. 
Nach der erfolgreichen Anmeldung mit der \acs{MFA} hat dieser die Möglichkeit auf die dann zur Verfügung stehenden Dienste vom 
Nginx Reverse Proxy Manager aus zuzugreifen.
\\Hat sich der Akteur mit dem Authentik Server verbunden, der aus dem dem eigentlichen Server (Authentik Server Core) und dem 
integrierten Außenposten (Embedded oupost) besteht, werden einkommende Anfragen an den Server-Containern und den Authentik Server Core 
und dem Embedded oupost geroutet. Der Authentik Core Server verarbeitet den Großteil der Logik von Authentik, wie \zB \acs{API}- und/ oder 
\acs{SSO}-Anfragen, während der Embedded outpost die Verwendung von Proxy-Anbietern ermöglicht, ohne dass eine separate Außenstelle 
eingerichtet werden muss. Der Hintergrundarbeiter (Background Worker) führt Hintergrundaufgaben aus, wie das Senden von E-Mails, 
oder Benachrichtigen von Ereignisses und alles, was im Frontend sichtbar ist. Authentik nutzt PostgreSQL, um alle seiner 
Konfigurationen und Daten zu speichern. Redis wird als Message-Queue und Cache verwendet.