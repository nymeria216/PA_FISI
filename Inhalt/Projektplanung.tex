% !TEX root = ../Projektdokumentation.tex
\section{Projektplanung} 
\label{sec:Projektplanung}


\subsection{Projektphasen}
\label{sec:Projektphasen}

Das Projekt findet in zwei Wochen unter maximal vier Stunden Arbeitszeit pro Tag statt, sodass zur gleichen 
Zeit die anderen Aufgaben und das Brainstorming für dieses Projekt erledigt werden können.
\\Die im Projektantrag festgelegten Projektphasen lassen sich chronologisch in die 5-stündige Planungsphase, 
die 16-stündige Implementierungsphase, die 6-stündige Testphase, die 4-stündige Phase, in der die 
Einführung und Übergabe erfolgt und die 7-stündige Dokumentation mit der 2-stündigen Pufferzeit im Anschluss.
\\Dabei änderte sich die Zeit in der Implementierungsphase von 14 auf 16 Stunden, die Zeit in der 
Dokumentation von 6 auf 7 Stunden und die Pufferzeit, welche hinzugefügt wurde, um zeitliche Konflikte mit 
den anderen Phasen zu vermeiden.

\paragraph{Beispiel}
Tabelle~\ref{tab:Zeitplanung} zeigt ein Beispiel für eine grobe Zeitplanung.
\tabelle{Zeitplanung}{tab:Zeitplanung}{ZeitplanungKurz}\\
Eine detailliertere Zeitplanung findet sich im \Anhang{app:Zeitplanung}.


\subsection{Abweichungen vom Projektantrag}
\label{sec:AbweichungenProjektantrag}
Die im Projektantrag mit ''in Auflage genehmigt'' Inhalte, erfordern Änderungen in der Projektdokumentation.
\\Die ursprüngliche Zeitplanung von 35 Stunden änderte sich auf 40 Stunden.
\\Zu Beginn erfolgen die Änderungen in der Zeitplanung, im Schritt Planungsphase, in welcher der Punkt ''Klärung der 
Projektziele'' in den Start dieser Phase geschoben werden.
\\Des Weiteren erfolgt in der Zeitplanung, Dokumentation, in welchem geplant keine Entwicklerdokumentation 
stattfindet, sondern gewünscht eine Benutzerdokumentation.
\\Die technische Umsetzung von Sitecars sollte in der Implementierungsphase nach der Auswahl einer MFA-Lösung 
sich ereignen, was bedingt der Nutzwertanalyse auf der Seite 6 in Kapitel 3 Analysephase, Punkt 3.3 
Nutzwertanalyse, sich auf Authelia änderte.


\subsection{Ressourcenplanung}
\label{sec:Ressourcenplanung}

\begin{itemize}
	\item Detaillierte Planung der benötigten Ressourcen (Hard-/Software, Räumlichkeiten \usw).
	\item \Ggfs sind auch personelle Ressourcen einzuplanen (\zB unterstützende Mitarbeiter).
	\item Hinweis: Häufig werden hier Ressourcen vergessen, die als selbstverständlich angesehen werden (\zB PC, Büro). 
\end{itemize}


\subsection{Entwicklungsprozess}
\label{sec:Entwicklungsprozess}
\begin{itemize}
	\item Welcher Entwicklungsprozess wird bei der Bearbeitung des Projekts verfolgt (\zB Wasserfall, agiler Prozess)?
\end{itemize}
