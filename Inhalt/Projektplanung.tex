% !TEX root = ../Projektdokumentation.tex
\section{Projektplanung} 
\label{sec:Projektplanung}


\subsection{Projektphasen}
\label{sec:Projektphasen}
Die im Projektantrag festgelegten Projektphasen lassen sich chronologisch in die 5-stündige Planungsphase, 
die 16-stündige Implementierungsphase, die 6-stündige Testphase, die 4-stündige Phase einteilen, in der die 
Einführung und Übergabe erfolgt und die 7-stündige Dokumentation im Anschluss.
Dabei änderte sich die Zeit in der Implementierungsphase von 14 auf 16 Stunden, die Zeit in der 
Dokumentation von 6 auf 7 Stunden.
\\Das Projekt findet in zwei Wochen, vom 30.10.2023 bis zum 10.11.2023 statt. Grund dafür ist ein Feiertag 
und ein möglicher entstehender Krankheitsfall.

Tabelle~\ref{tab:Zeitplanung} zeigt ein Beispiel für eine grobe Zeitplanung.
\tabelle{Zeitplanung}{tab:Zeitplanung}{ZeitplanungKurz}\\
Eine detailliertere Zeitplanung findet sich im \Anhang{app:Zeitplanung}.


\subsection{Abweichungen vom Projektantrag}
\label{sec:AbweichungenProjektantrag}
Die im Projektantrag mit in \textit{Auflage genehmigt}en Inhalte, erforderten Änderungen in der 
Projektdokumentation. Einige Änderungen sind im Kapitel \ref*{sec:Entwurfsphase} 
\nameref{sec:Authentifizierungs-Tool} einsehbar. Änderungen:
\begin{itemize} [label=--]
	\item gesamte Zeitplanung - von 35 auf \textit{40} Stunden gestreckt
	\item Stundenplanung - Planungsphase von 5 auf \textit{8} Stunden
	\item Stundenplanung - Implementierungsphase von 14 auf \textit{16} Stunden
	\item Stundenplanung - Tesphase von 6 auf \textit{4} Stunden
	\item Stundenplanung - Dokumentation von 8 auf \textit{6} Stunden
	\item Zeitplanung/ Planungsphase - \textit{Klärung der Projektziele} nach \textit{vorn} der Phase geschoben
	\item Zeitplanung/ Dokumentation - \textit{Benutzerdokumentation} statt Entwicklerdokumentation
	\item Implementierungsphase/ Installation und Konfiguration von Sitecars - \textit{Authentik} statt Sitecars
\end{itemize}


\subsection{Ressourcenplanung}
\label{sec:Ressourcenplanung}

\subsubsection{Sachmittelplanung}
\label{sec:Sachmittelplanung}
Um die Umsetzung des Projektes zu ermöglichen, wurden folgende Hard- und Software verwendet:
\begin{itemize} [label=--]
	\item Notebook - Lenovo ThinkPad T15 Gen 1 (für die Entwicklung)
	\item Betriebssystem - Microsoft Windows 10 Enterprise auf dem Lenovo-Notebook
	\item iPhone 12 - iOS 17.0.3 (zum Testen des Einmalpassworts)
	\item Microsoft Authenticator (auf dem iPhone 12 vorinstalliert)
	\item IDE - Visual Studio Code 1.83.1 (user setup)
	\item Docker-Containerisierung aufder OVHCloud-Infrastruktur in einer Linux-Umgebung
	\item OVHCloud-Instanz - firewall\_instance\_dev (flavor name: b2-7)
	\item OVHCloud-Instanz - tal\_cloud\_infra (flavor name: r2-60)
	\item OVHCloud-Instanz - rev\_prox-dev (flavor name: b2-15)
\end{itemize}

\subsubsection{Personalplanung}
\label{sec:Personalplanung}
% \paragraph{Tabelle}
Tabelle~\ref{tab:Personalplanung} zeigt die Personalplanung des Projektes.
\tabelle{Personalplanung}{tab:Personalplanung}{Personalplanung.tex}

\subsubsection{Ablaufplanung und Meilensteine}
\label{sec:Ablaufplaung und Meilensteine}
Die Ablaufplanung ist mit einem \nameref{app:Gantt} im Kapitel \ref{app:Gantt} des \nameref{sec:Anhang}s dargestellt. 
Dabei stellt die Farbe pink \textit{keine Arbeitszeit} an dem Tag dar, die Farbe grün eine \textit{reine Abrbeitszeit} 
von 8 Stunden am Tag und die hellblaue Farbe einen \textit{halben Arbeitstag} von 4 Stunden. Die jeweiligen Phasen 
werden mit Meilensteinen abgeschlossen. Die erste Phase startet am Mittwoch, den 01.11.2023, während die letzte 
mit ihrem Meilenstein am 08.11.2023 endet. Dabei erfolgt die reguläre Arbeitszeit in einer normalen Arbeitswoche 
von Montag bis Freitag.

\subsection{Entwicklungsprozess}
\label{sec:Entwicklungsprozess}
Das Projekt unterteilt sich in einem überschaubaren, zeitlich und inhaltlich begrenzten Entwicklungsprozess 
mit gesondert begrenzten Phasen, die nach- und voneinander aufbauen. So wird eine Sicherstellung der Schritt- 
für Schritt-Fertigstellung der jeweiligen Phasen und Übersicht garantiert. Bedingt dessen, dass das Projekt klare Anforderungen 
hat, welche umfassend mit vorhersehbaren Bedingungen formuliert sind und einen linearen Fortschritt ermöglicht, braucht 
die agile Methode nicht verwendet zu werden. Auch ist der Umfang des Projektes zeitlich abgegrenzt und besitzt eine 
strukturierte Vorgehensweise mit einer geringen Interaktion mit dem Entwicklerteam und dem Projektleiter als Kunden.

\subsection{Anforderungsanalyse}
\label{sec:Anforderungsanalyse}

\subsubsection{Funktional}
\label{sec:Funktional}
Authenik muss folgende funktionale Anforderungen erfüllen:
\begin{itemize} [label=--]
	\item problemlose Registrierung und einfacher Login
	\item nachvollziehbare Schritte während der Anmeldung bei der Eingabe des Einmalpassworte
	\item übersichtliche Darstellung und Verwaltung der einzuloggenden Nutzer :innen
	\item Login als Admin und durchschnittlicher User
	\item vor jeden konfigurierten Service schalten und nach Eingabe der Nutzerdaten freigeben
	\item muss sicher sein und Zugriff von Dritten verweigern
	\item kompatibel mit einer Authenticator-Software
	\item schnelle Eingabe der Nutzerdaten
\end{itemize}

\subsubsection{Nicht-Funktional}
\label{sec:Nicht-Funktional}
Authentik muss folgende nicht-funktionale Anforderungen erfüllen:
\begin{itemize} [label=--]
	\item Skalierbarkeit der Services
	\item das Überleiten auf den nächsten Service sollte in weniger als 2 Sekunden geladen haben
	\item Enthalten des Brandings vond er Deloitte Wirtschaftsprüfungsgesellschaft GmbH
	\item übersichtliche, englischsprachiges Willkommens-Display
\end{itemize}
	