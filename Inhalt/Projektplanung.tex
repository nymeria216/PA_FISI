% !TEX root = ../Projektdokumentation.tex
\section{Projektplanung} 
\label{sec:Projektplanung}


\subsection{Projektphasen}
\label{sec:Projektphasen}
Die im Projektantrag festgelegten Projektphasen lassen sich chronologisch in die 8-stündige Planungsphase, 
die 16-stündige Implementierungsphase, die 4-stündige Testphase, die 4-stündige Phase der Einführung und Übergabe, 
und die 8-stündige Dokumentation einteilen.
Dabei änderte sich die Zeit in der Implementierungsphase von 14 auf 16 Stunden, die Zeit in der 
Dokumentation von 6 auf 8 Stunden.
\\Das Projekt findet in zwei Wochen, vom 30.10.2023 bis zum 10.11.2023 statt.

Tabelle~\ref{tab:Zeitplanung} \nameref{tab:Zeitplanung} zeigt ein Beispiel für eine grobe Zeitplanung. 
Eine detailliertere Zeitplanung befindet sich im \nameref{sec:Anhang} \ref{app:Zeitplanung}.
\tabelle{Zeitplanung}{tab:Zeitplanung}{ZeitplanungKurz}


\subsection{Abweichungen vom Projektantrag}
\label{sec:AbweichungenProjektantrag}
Die im Projektantrag mit in \textit{Auflage genehmigt}en Inhalte, erforderten Änderungen in der 
Projektdokumentation. Einige Änderungen sind im Kapitel \ref*{sec:Entwurfsphase} 
\nameref{sec:Authentifizierungs-Tool} einsehbar. 
\\Erforderliche Änderungen:
\begin{itemize} [label=--]
	\item gesamte Zeitplanung - von 35 auf \textit{40} Stunden gestreckt
	\\Folgen der Zeitänderungen:
	\begin{itemize}
		\item Stundenplanung - Planungsphase von 5 auf \textit{8} Stunden
		\item Stundenplanung - Implementierungsphase von 14 auf \textit{16} Stunden
		\item Stundenplanung - Tesphase von 6 auf \textit{4} Stunden
		\item Stundenplanung - Dokumentation von 8 auf \textit{6} Stunden 
	\end{itemize}
	\item Zeitplanung/ Planungsphase - \textit{Klärung der Projektziele} nach \textit{vorn} der Phase geschoben
	\item Zeitplanung/ Dokumentation - \textit{Benutzerdokumentation} statt Entwicklerdokumentation
\end{itemize}

Nicht erforderliche Änderungen:
\begin{itemize} [label=--]
	\item Implementierungsphase/ Installation und Konfiguration von Sitecars - \textit{Authentik} statt Sitecars
	\\Eine genaue Schilderung, weshalb sich für Authentik anstatt Sitecars entschieden wurde, befindet sich im 
	Kapitel \ref*{sec:Authentifizierungs-Tool} \nameref{sec:Authentifizierungs-Tool}
\end{itemize}


\subsection{Ressourcenplanung}
\label{sec:Ressourcenplanung}

\subsubsection{Sachmittelplanung}
\label{sec:Sachmittelplanung}
Um die Umsetzung des Projektes zu ermöglichen, wurden folgende Hard- und Software verwendet:
\begin{itemize} [label=--]
	\item Notebook - Lenovo ThinkPad T15 Gen 1 (für die Entwicklung)
	\item Betriebssystem - Microsoft Windows 10 Enterprise auf dem Lenovo-Notebook
	\item iPhone 12 - iOS 17.0.3 (zum Testen des Einmalpassworts)
	\item Microsoft Authenticator (auf dem iPhone 12 vorinstalliert)
	\item \acs{IDE} - Visual Studio Code 1.83.1 (Benutzereinstellung)
	\item Docker-Containerisierung auf der OVHCloud-Infrastruktur in einer Linux-Umgebung mit Ubuntu 22.04
	\item OVHCloud-Instanz - firewall\_instance\_dev
	\item OVHCloud-Instanz - tal\_cloud\_infra
	\item OVHCloud-Instanz - rev\_prox-dev
	\item Google Chrome
\end{itemize}
Die von OVHCloud gestellten Instanzen sind Cloud-Instanzen mit Linux-Umgebungen, auf denen 
die Containerisierungen laufen und nicht von den Benutzerkonten aus betrieben werden. Dabei wurde im Kapitel 
\ref{sec:Projektschnittstellen} \nameref{sec:Projektschnittstellen} detaillierter auf die Namen der jeweiligen 
Instanzen eingegangen.


\subsubsection{Personalplanung}
\label{sec:Personalplanung}
% \paragraph{Tabelle}
Tabelle~\ref{tab:Personalplanung} zeigt die \nameref{tab:Personalplanung} des Projektes.
\tabelle{Personalplanung}{tab:Personalplanung}{Personalplanung.tex}\\
In diesem Projekt arbeitet die Autorin Melissa Futtig 40 Stunden die Woche, während der Projektmanager 
Edgar Johann Kapler 2 Stunden in das Projekt für das Testen der Anwendung investiert und in dem 
Übergabeprozess, da ihm das Ergebnis des Projektes schlussendlich vorgelegt wird. Der duale Student 
Birk Spinn unterstützt bei der Implementierung und dem Testen der Anwendung und benötigt für diese 
Aufgaben insgesamt 3 Stunden.

\subsubsection{Ablaufplanung und Meilensteine}
\label{sec:Ablaufplaung und Meilensteine}
Die Ablaufplanung ist mit einem \nameref{app:Gantt} im Kapitel \ref{app:Gantt} des \nameref{sec:Anhang}s dargestellt. 
Dabei stellt die Farbe pink \textit{keine Arbeitszeit}, die Farbe grün eine \textit{reine Abrbeitszeit} 
von 8 Stunden am Tag und die hellblaue Farbe einen \textit{halben Arbeitstag} von 4 Stunden dar. Die jeweiligen Phasen 
werden mit Meilensteinen abgeschlossen. Die erste Phase startet am Mittwoch, den 01.11.2023, während die letzte 
mit ihrem Meilenstein am 08.11.2023 endet. Dabei erfolgt die reguläre Arbeitszeit in einer normalen Arbeitswoche 
von Montag bis Freitag.

\subsection{Entwicklungsprozess}
\label{sec:Entwicklungsprozess}
Das Projekt unterteilt sich in einem überschaubaren, zeitlich und inhaltlich begrenzten Entwicklungsprozess 
mit gesondert eingeteilten Phasen, die nach- und voneinander aufbauen. So wird eine Sicherstellung der Schritt- 
für Schritt-Fertigstellung der jeweiligen Phasen und Übersicht garantiert. Bedingt dessen, dass das Projekt klare Anforderungen 
hat, welche umfassend mit vorhersehbaren Bedingungen formuliert sind und einen linearen Fortschritt ermöglicht, braucht 
die agile Methode nicht verwendet zu werden. Auch ist der Umfang des Projektes zeitlich eingegrenzt und besitzt eine 
strukturierte Vorgehensweise mit einer geringen Interaktion zum und mit dem Entwicklerteam und des Projektleiters als Kunden.

\subsection{Anforderungsanalyse}
\label{sec:Anforderungsanalyse}

\subsubsection{Funktional}
\label{sec:Funktional}
Die MFA-Lösung muss folgende funktionale Anforderungen erfüllen:
\begin{itemize} [label=--]
	\item Benutzerfreundliche Registrierung und Login
	\item Klare Anleitung für die Einmalpasswort-Eingabe
	\item Übersichtliche Verwaltung von Nutzern
	\item Einloggen als Admin und Benutzer
	\item Dienste nach Benutzerdateneingabe aktivieren
	\item Schutz vor Drittzugriff, sichere Lösung
	\item Nutzung mit Authentikator-Software
\end{itemize}

\subsubsection{Nicht-Funktional}
\label{sec:Nicht-Funktional}
Authentik muss folgende nicht-funktionale Anforderungen erfüllen:
\begin{itemize} [label=--]
	\item Skalierbarkeit der Dienste
	\item Übergang zu Diensten in bis zu 2 Sekunden
	\item Übersichtliches Willkommensdisplay (Englisch)
\end{itemize}
	