% !TEX root = ../Projektdokumentation.tex
\section{Projektplanung} 
\label{sec:Projektplanung}


\subsection{Projektphasen}
\label{sec:Projektphasen}
Die im Projektantrag festgelegten Projektphasen lassen sich chronologisch in die 5-stündige Planungsphase, 
die 16-stündige Implementierungsphase, die 6-stündige Testphase, die 4-stündige Phase einteilen, in der die 
Einführung und Übergabe erfolgt und die 7-stündige Dokumentation mit der 2-stündigen Pufferzeit im Anschluss.
Dabei änderte sich die Zeit in der Implementierungsphase von 14 auf 16 Stunden, die Zeit in der 
Dokumentation von 6 auf 7 Stunden und die Pufferzeit von 2 Stunden, welche hinzugefügt wurde, um zeitliche Konflikte mit 
den anderen Phasen zu vermeiden.
\\Das Projekt findet in zwei Wochen, vom 30.10.2023 bis zum 10.11.2023 statt. Grund dafür ist ein Feiertag, 
sodass zweimal 8 Stunden entfallen und die 2 Stunden Pufferzeit in den zwei Wochen während der Arbeitszeit 
erfolgen können. Zusätzlich ist ein eventueller Krankheitsfall mit einkalkuliert.

Tabelle~\ref{tab:Zeitplanung} zeigt ein Beispiel für eine grobe Zeitplanung.
\tabelle{Zeitplanung}{tab:Zeitplanung}{ZeitplanungKurz}\\
Eine detailliertere Zeitplanung findet sich im \Anhang{app:Zeitplanung}.


\subsection{Abweichungen vom Projektantrag}
\label{sec:AbweichungenProjektantrag}

Die im Projektantrag mit ''in Auflage genehmigten'' Inhalte, erfordern Änderungen in der Projektdokumentation. 
\\Die ursprüngliche Zeitplanung von 35 Stunden streckt sich auf 40 Stunden.
Im Anschluss passieren Änderungen in der Zeitplanung, im Schritt der Planungsphase, in welcher der Punkt ''Klärung der 
Projektziele'' in den Start dieser Phase geschoben wird. Des Weiteren erfolgt in der Zeitplanung, der 
Dokumentation, in welchem geplant keine Entwicklerdokumentation stattfindet, sondern gewünscht eine 
Benutzerdokumentation. Die technische Umsetzung von Sitecars sollte sich in der Implementierungsphase 
nach der Auswahl einer MFA-Lösung ereignen, was bedingt der Nutzwertanalyse auf der Seite 
\pageref{sec:Authentifizierungs-Tool} im Kapitel~\ref{sec:Entwurfsphase}: \nameref{sec:Authentifizierungs-Tool} 
auf Authentik änderte.


\subsection{Ressourcenplanung}
\label{sec:Ressourcenplanung}

\subsubsection{Sachmittelplanung}
\label{sec:Sachmittelplanung}
Um die Umsetzung des Projektes zu ermöglichen, wurden folgende Hard- und Software verwendet:
\begin{itemize} [label=--]
	\item Notebook - Lenovo ThinkPad T15 Gen 1 (für die Entwicklung)
	\item Betriebssystem - Microsoft Windows 10 Enterprise auf dem Lenovo-Notebook
	\item iPhone 12 - iOS 17.0.3 (zum Testen des Einmalpassworts)
	\item Microsoft Authenticator (auf dem iPhone 12 vorinstalliert)
	\item IDE - Visual Studio Code 1.83.1 (user setup)
	\item Docker-Containerisierung aufder OVHCloud-Infrastruktur in einer Linux-Umgebung
	\item OVHCloud-Instanz - firewall\_instance\_dev (flavor name: b2-7)
	\item OVHCloud-Instanz - tal\_cloud\_infra (flavor name: r2-60)
	\item OVHCloud-Instanz - rev\_prox-dev (flavor name: b2-15)
\end{itemize}

\subsubsection{Personalplanung}
\label{sec:Personalplanung}
% \paragraph{Tabelle}
Tabelle~\ref{tab:Personalplanung} zeigt die Personalplanung des Projektes.
\tabelle{Personalplanung}{tab:Personalplanung}{Personalplanung.tex}

\subsubsection{Ablaufplanung}
\label{sec:Ablaufplaung und Meilensteine}
Die Ablaufplanung ist mittels eines \nameref{app:Gantt}-Diagramms im Kapitel \ref{app:Gantt} des \nameref{sec:Anhang}s 
auf der Seite \pageref{app:Gantt} dargestellt.

\subsection{Entwicklungsprozess}
\label{sec:Entwicklungsprozess}
Das Projekt unterteilt sich in einem überschaubaren, zeitlich und inhaltlich begrenzten Entwicklungsprozess 
mit einzelnene begrenzten Phasen, die nach- und voneinander aufbauen. So wird eine Sicherstellung der Schritt 
für Schritt-Beendingung der jeweiligen Phasen und Übersicht gewährleistet. Bedingt dessen, dass in diesem Projekt 
wenig Projektteilnehmer: innen zur Verfügung stehen, kann auf die agile Methodik verzichtet werden.

\subsection{Anforderungsanalyse}
\label{sec:Anforderungsanalyse}

\subsubsection{Funktional}
\label{sec:Funktional}
Authenik muss folgende funktionale Anforderungen erfüllen:
\begin{itemize} [label=--]
	\item problemlose Registrierung und einfacher Login
	\item nachvollziehbare Schritte während der Anmeldung bei der Eingabe des Einmalpassworte
	\item übersichtliche Darstellung und Verwaltung der einzuloggenden Nutzer :innen
	\item Login als Admin und durchschnittlicher User
	\item vor jeden konfigurierten Service schalten und nach Eingabe der Nutzerdaten freigeben
	\item muss sicher sein und Zugriff von Dritten verweigern
	\item kompatibel mit einer Authenticator-Software
	\item schnelle Eingabe der Nutzerdaten
\end{itemize}

\subsubsection{Nicht-Funktional}
\label{sec:Nicht-Funktional}
\begin{itemize} [label=--]
	\item Skalierbarkeit der Services
	\item das Überleiten auf den nächsten Service sollte in weniger als 2 Sekunden geladen haben
	\item Enthalten des Brandings vond er Deloitte Wirtschaftsprüfungsgesellschaft GmbH
	\item übersichtliche, englischsprachiges Willkommens-Display
\end{itemize}
	