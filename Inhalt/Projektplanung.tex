% !TEX root = ../Projektdokumentation.tex
\section{Projektplanung} 
\label{sec:Projektplanung}


\subsection{Projektphasen}
\label{sec:Projektphasen}

Das Projekt findet in zwei Wochen in vier Stunden Arbeitszeit pro Tag statt, sodass Pufferzeit und 
Krankheit eingerechnet werden können.
\\Die im Projektantrag festgelegten Projektphasen lassen sich chronologisch in die 5-stündige Planungsphase, 
die 16-stündige Implementierungsphase, die 6-stündige Testphase, die 4-stündige Phase, in der die 
Einführung und Übergabe erfolgt und die 7-stündige Dokumentation mit der 2-stündigen Pufferzeit im Anschluss.
Dabei änderte sich die Zeit in der Implementierungsphase von 14 auf 16 Stunden, die Zeit in der 
Dokumentation von 6 auf 7 Stunden und die Pufferzeit, welche hinzugefügt wurde, um zeitliche Konflikte mit 
den anderen Phasen zu vermeiden.

Tabelle~\ref{tab:Zeitplanung} zeigt ein Beispiel für eine grobe Zeitplanung.
\tabelle{Zeitplanung}{tab:Zeitplanung}{ZeitplanungKurz}\\
Eine detailliertere Zeitplanung findet sich im \Anhang{app:Zeitplanung}.


\subsection{Abweichungen vom Projektantrag}
\label{sec:AbweichungenProjektantrag}

Die im Projektantrag mit ''in Auflage genehmigten'' Inhalte, erfordern Änderungen in der Projektdokumentation.
\\Die ursprüngliche Zeitplanung von 35 Stunden streckt sich auf 40 Stunden.
\\Im Anschluss passieren Änderungen in der Zeitplanung, im Schritt der Planungsphase, in welcher der Punkt ''Klärung der 
Projektziele'' in den Start dieser Phase geschoben werden.
\\Des Weiteren erfolgt in der Zeitplanung, der Dokumentation, in welchem geplant keine Entwicklerdokumentation 
stattfindet, sondern gewünscht eine Benutzerdokumentation.
\\Die technische Umsetzung von Sitecars sollte sich in der Implementierungsphase nach der Auswahl einer MFA-Lösung 
ereignen, was bedingt der Nutzwertanalyse auf der Seite 6 in Kapitel 3 Analysephase, Punkt 3.3 Nutzwertanalyse, 
sich auf Authelia änderte.


\subsection{Ressourcenplanung}
\label{sec:Ressourcenplanung}

\subsubsection{Sachmittelplanung}
\label{sec:Sachmittelplanung}
Um die Umsetzung des Projektes zu ermöglichen, wurden folgende Hard- und Software verwendet:
\begin{itemize}
	\item Notebook - Lenovo ThinkPad T15 Gen 1 (für die Entwicklung)
	\item Betriebssystem - Microsoft Windows 10 Enterprise auf dem Lenovo-Notebook
	\item IDE - Visual Studio Code 1.83.1 (user setup)
	\item Docker-Containerisierung der OVHCloud-Infrastruktur in einer Linux-Umgebung
	\item OVHCloud-Instanz - firewall\_instance\_dev (flavor name: b2-7)
	\item OVHCloud-Instanz - tal\_cloud\_infra (flavor name: r2-60)
	\item OVHCloud-Instanz - rev\_prox-dev (flavor name: b2-15)
\end{itemize}

\subsubsection{Personalplanung}
\label{sec:Personalplanung}
% \paragraph{Tabelle}
Tabelle~\ref{tab:Personalplanung} zeigt die Personalplanung des Projektes.
\tabelle{Personalplanung}{tab:Personalplanung}{Personalplanung.tex}

\subsection{Entwicklungsprozess}
\label{sec:Entwicklungsprozess}
Das Projekt unterteilt sich in einem überschaubaren, zeitlich und inhaltlich begrenzten Entwicklungsprozess 
mit einzelnene begrenzten Phasen, die nach- und voneinander aufbauen. So wird eine Sicherstellung der Schritt 
für Schritt-Beendingung der jeweiligen Phasen und Übersicht gewährleistet. Bedingt dessen, dass in diesem Projekt 
wenig Projektteilnehmer: innen zur Verfügung stehen, kann auf die agile Methodik verzichtet werden.
