% !TEX root = ../Projektdokumentation.tex
\section{Projektplanung} 
\label{sec:Projektplanung}

\subsection{Projektphasen}
\label{sec:Projektphasen}
Die im Projektantrag festgelegten Projektphasen lassen sich chronologisch in die 8-stündige Planungsphase, 
die 16-stündige Implementierungsphase, die 4-stündige Testphase, die 4-stündige Phase der Einführung und Übergabe 
und die 8-stündige Dokumentation einteilen.
Dabei änderte sich die Zeit in der Implementierungsphase von 14 auf 16 Stunden, die Zeit in der 
Dokumentation von 6 auf 8 Stunden.
\\Das Projekt findet in zwei Wochen, vom 30.10.2023 bis zum 10.11.2023 statt.

Tabelle~\ref{tab:Zeitplanung} \nameref{tab:Zeitplanung} zeigt ein Beispiel für eine grobe Zeitplanung. 
Eine detailliertere Zeitplanung befindet sich im \nameref{sec:Anhang} \nameref{tab:Zeitplanung}.
\tabelle{Zeitplanung}{tab:Zeitplanung}{ZeitplanungKurz}


\subsection{Authentifizierungs-Tool}
\label{sec:Authentifizierungs-Tool}
Anhand der Entscheidungsmatrix in Tabelle~\ref{tab:Nutzwert} wurde Authentik ausgewählt. 
\\Die Gewichtung bildet sich durch den Gesamtwert von 100 Wertungspunkten mit einem Maximalwert von 25 und Mindestwert von 10 Punkten. 
Grund dafür sind die unterschiedlichen Eigenschaften, welche in der Entwicklung und bei der Auswahl der 
Authentifizierungsmethode jeweils verschiedene Rollen spielen und folglich daraus evaluiert werden. Nach der Zuordnung 
und Addierung der Punkte bei den vier Authentifizierungs-Tools, wird die Gesamtheit aller Punkte eines Produktes durch den 
Wert 100 dividiert und das Endergebnis ausgerechnet. Dabei hat die Einhaltung der Sicherheit Priorität und erhält den 
Maximalwert von 25 Punkten, aufgrund dessen, dass Dritten der Zugriff auf die jeweiligen Dienste mit kunden- und 
firmeninternen Daten der Cloud-Infrastruktur verweigert werden sollte. Die \acs{MFA} wird mit 20 Punkten bewertet, weil 
dass das Ziel des Projektes ist. Die Benutzerfreundlichkeit, das Implementieren mit Open-Source und die Skalierbarkeit 
erhalten 15 Punkte, weil jeder User schnell und einfach auf den jeweiligen Dienst zugreifen muss. 
Für dieses Projekt ist es des Weiteren wichtig, ein Tool zu implementieren, was den zeitlichen Rahmen nicht überschreitet und die 
Gesamtheit der Einführung zu komplex gestaltet. Die Skalierbarkeit hat für das Projekt eine durchschnittliche Relevanz, da die 
Möglichkeit bestehen soll, Dienste hinzuzufügen oder rauszunehmen. Am wenigsten bedeutend sind die Erfahrungswerte, welche 
gleichermaßen nicht zu unterschätzen sind, weil eine Community über das Produkt bei der Entwicklung unterstützend wirkend kann.
\\Dabei erhält Authentik den größten Nutzwert mit 17,05 Punkten und schneidet vor Authelia, Microsoft Azure AD und Sitecar am besten ab. 
Aus diesem Grund wird sich für Authentik anstatt für Sitecar entschieden, da besonders die Implementierungsdauer und -komplexität 
bei Sitecar den Rahmen des Projektes sprengen würde.
\tabelle{Entscheidungsmatrix}{tab:Nutzwert}{Nutzwert.tex}
\\Dabei steht \textit{Mic. Az.} für Microsoft Azure.
\\Die zu resultierende Zielplattform definiert sich über die Benutzerfreundlichkeit, die Sicherheit und dem Fokus auf der Interaktion 
mit \acs{OIDC} und \acs{LDAP}, welche zukünftig für das Projekt vorgesehen sind, sowie der Implementierung vom Open-Source, dem Arbeiten mit MFA, 
der Skalierbarkeit und den Erfahrungswerten von anderen Entwicklern mit dem entsprechenden Produkt. Das Resultat wird im Kapitel 
\ref{sec:Authentifizierungs-Tool}: \nameref{sec:Authentifizierungs-Tool} der dargestellten~\nameref{tab:Nutzwert} sichtbar.

\subsection{Ressourcenplanung}
\label{sec:Ressourcenplanung}

\subsubsection{Sachmittelplanung}
\label{sec:Sachmittelplanung}
Um die Umsetzung des Projektes zu ermöglichen, wurden folgende Hard- und Software verwendet:
\begin{itemize} [label=--]
	\item Notebook - Lenovo ThinkPad T15 Gen 1 (für die Entwicklung)
	\item Betriebssystem - Microsoft Windows 10 Enterprise auf dem Lenovo-Notebook
	\item iPhone 12 - iOS 17.0.3 (zum Testen des Einmalpassworts)
	\item Microsoft Authenticator (auf dem iPhone 12 vorinstalliert)
	\item \acs{IDE} - Visual Studio Code 1.83.1 (Benutzereinstellung)
	\item Docker-Containerisierung auf der OVH-Cloud-Infrastruktur in einer Linux-Umgebung mit Ubuntu 22.04
	\item OVH-Cloud-Instanz - firewall\_instance\_dev
	\item OVH-Cloud-Instanz - tal\_cloud\_infra
	\item OVH-Cloud-Instanz - rev\_prox-dev
	\item Google Chrome - Browser
\end{itemize}
Die von OVH-Cloud gestellten Instanzen sind Cloud-Instanzen mit Linux-Umgebungen, auf denen 
die Containerisierungen laufen. Dabei wurde im Kapitel \ref{sec:Projektschnittstellen} 
\nameref{sec:Projektschnittstellen} detaillierter auf die Namen der jeweiligen Instanzen eingegangen.

\subsubsection{Personalplanung}
\label{sec:Personalplanung}
% \paragraph{Tabelle}
Tabelle~\ref{tab:Personalplanung} zeigt die \nameref{tab:Personalplanung} des Projektes.
\tabelle{Personalplanung}{tab:Personalplanung}{Personalplanung.tex}\\
In diesem Projekt arbeitet die Autorin Melissa Futtig 40 Stunden am Projekt, während der Projektmanager 
Edgar Johann Kapler 2 Stunden für das Testen der Anwendung und dem Übergabeprozess aufwendet, da ihm 
das Ergebnis des Projektes schlussendlich übergeben wird. Der duale Student Birk Spinn unterstützt bei 
der Implementierung und dem Testen der Anwendung und benötigt für diese Aufgaben insgesamt 3 Stunden.

\subsection{Kostenplanung}
\label{sec:Kostenplanung}
Die Kosten für die Durchführung des Projekts setzen sich aus den Personal- und Ressourcenkosten zusammen.   
Das Brutto-Einkommen eines Auszubildenden im 3. Lehrjahr im Fachbereich Fachinformatik bei der Deloitte Wirtschaftsprüfungsgesellschaft GmbH 
beträgt \eur{1400,00} pro Monat. Zu den \eur{1400,00} kommen \eur{1090,00} Versicherungskosten dazu, die die Deloitte bezahlt. 
Die reguläre Arbeitszeit in einer normalen Arbeitswoche von Montag bis Freitag beträgt 40 Stunden, 8 Stunden am Tag und 
220 Arbeitstage im Jahr. Die 220 Tage entstehen, da von den 365 Tagen eines Kalenderjahres 104 Wochenendtage, 11 Feiertage und 30 Urlaubstage abgezogen.
Für die weiteren Mitarbeiter werden pauschale Beträge zur Berechnung des Stundensatzes genutzt. 
Duale Studenten werden pauschal mit \eur{15,00}, während die Manager mit \eur{75,00} pro Stunde berechnet werden. 
Bei jeweils beiden addiert sich die Summe der Ressourcenkosten auf.
Anhand der oben genannten Formel ergibt sich ein Stundenlohn von \eur{9,55}. Dieser wird aus dem Brutto-Einkommen des Auszubildenden berechnet.
\\Die Durchführungszeit des Projekts beträgt 40 Stunden. 
Die Anforderungen durch die Ressourcen, wie der Stromverbrauch, die zu verwendende Hardware und die Räumlichkeiten, sowie das Büromaterial, 
wie \zB der zu nutzende Monitor, die Peripheriegeräte (Maus, Tastatur, etc.) und das Mobiliar zusammen.
Der Mittelwert wird pauschal mit \eur{15,00} kalkuliert.  
\begin{eqnarray}
	8 \mbox{ h/Tag} \cdot 220 \mbox{ Tage/Jahr} = 1760 \mbox{ h/Jahr}\\
	\eur{2490}\mbox{/Monat} \cdot 12 \mbox{ Monate/Jahr} = \eur{29880} \mbox{/Jahr}\\
	\frac{\eur{29880} \mbox{/Jahr}}{1760 \mbox{ h/Jahr}} \approx \eur{16,98}\mbox{/h}
\end{eqnarray}
\tabelle{Kostenaufstellung}{tab:Kostenaufstellung}{Kostenaufstellung.tex}
Die Gesamtkosten, dargestellt in der Tabelle~\ref{tab:Kostenaufstellung} betragen \eur{1.561,88}.

\subsection{Wirtschaftlichkeitsanalyse}
\label{sec:Wirtschaftlichkeitsanalyse}
Durch die schon vorhandene Cloud-Infrastruktur des größeren Projektes entstehen keine weiteren Kosten. Bedingt dessen, dass die 
OVH-Cloud-Infrastruktur zu einem Fix-Preis pro Instanz gemietet wird und keine weiteren Instanzen für die Implementierung 
des Tochter-Projektes erforderlich sind, bleiben die Kosten unverändert. 
An Ressourcen wird zwar mehr \acs{CPU} und Rechenleistung verwendet, was allerdings nicht nach dem \cite{Wiki}-Prinzip 
berechnet wird, sondern bei OVH-Cloud pauschal nach Instanzpreis, sodass für das ''Tochter-Projekt'' keine weitere 
Kosten entstehen. In dem Pay-As-You-Go-Prinzip werden nur die Ressourcen bezahlt, die auch tatsächlich genutzt werden. 
Die wirklich zu entstehenden Kosten sind ausschließlich Personal- und Materialkosten, wobei letztere 
aus der Nutzung des Büromobiliars und der Strom- und Heizkosten zusammengefasst wird. 
\\Durch die Einführung der \acs{MFA}-Lösung in der Cloud-Infrastruktur werden besonders die Schutzziele der 
Einhaltung der Integrität und Vertraulichkeit der Daten auf den jeweiligen Diensten eingehalten. So wird die Sicherheit 
erheblich verbessert und trägt dazu bei, unbefugten Zugriff, Datenverluste und Betrug zu verhindern. Dieser Schutz 
vor Sicherheitsverletzungen kann erhebliche finanzielle Auswirkungen haben, da die Wiederherstellungskosten vermieden 
werden können. Zusätzlich ermöglicht die Implementierung, dass Passwortänderungen nur auf Anwendungsebene vorgenommen 
werden, wodurch weniger Zurücksetzungen erforderlich sind. Dies reduziert die Wahrscheinlichkeit von gleichen Passwörtern 
und verhindert dadurch den Bedarf an administrativem Support für Passwort-Resets, wodurch Zeit- und Kostenaufwände vermieden 
werden können. Durch die Implementierung kann den Nutzern ein sicherer und bequemerer Zugriff gewährleistet und die Kosten gesenkt werden. 
Die Einsicht erfolgt in der \nameref{sec:Kostenplanung} im Kapitel \ref{sec:Kostenplanung}.

\subsubsection{Ablaufplanung und Meilensteine}
\label{sec:Ablaufplaung und Meilensteine}
Die Ablaufplanung ist mit einem \nameref{app:Gantt} im Kapitel \ref{app:Gantt} des \nameref{sec:Anhang}s dargestellt. 
Dabei stellt die Farbe pink \textit{keine Arbeitszeit}, die Farbe grün eine \textit{reine Abrbeitszeit} 
von 8 Stunden am Tag und die hellblaue Farbe einen \textit{halben Arbeitstag} von 4 Stunden dar. Die jeweiligen Phasen 
werden mit Meilensteinen abgeschlossen. Die erste Phase startet am Mittwoch, den 01.11.2023, während die letzte 
mit ihrem Meilenstein am 08.11.2023 endet. Dabei erfolgt die reguläre Arbeitszeit in einer normalen Arbeitswoche 
von Montag bis Freitag.

\subsection{Amortisationsdauer}
\label{sec:Amortisationsdauer}
Die Amortisation beschleunigt sich durch die Verwendung von Docker, GitLab und Authentik. 
Grund dafür ist, dass diese Plattformen Open-Source sind und kostenlos genutzt werden können, was zu einer Reduzierung der  
Gesamtbetriebskosten (Total Cost of Ownerships (\acs{TCO})) führt, da Lizenzkosten eingespart werden können.  
% Des Weiteren ermöglicht Docker eine Arbeitszeitersparnis durch die einfache Bereitstellung und Verwaltung von Diensten, was die Arbeitszeit für die 
% Einrichtung und Wartung von Umgebungen verkürzt.
Eine Zeitersparnis entsteht durch das nicht wiederholte Eingeben der Benutzerdaten, zumal sich durch Authentik nur 
einmal eingeloggt werden muss und dadurch ein Zugriff auf alle Applikationen erfolgt.
\\Bei der Einsparung von einer Minute am Tag für jeden der 7 Anwender, nach der dualen Ausbildung und/ oder dem Studium und 
220 Arbeitstagen im Jahr, ergibt sich eine gesamte Zeiteinsparung von 
\begin{eqnarray}
7 \cdot 220 \mbox{ Tage/Jahr} \cdot 1 \mbox{ min/Tag} = 1540 \mbox{ min/Jahr} \approx 25,67 \mbox{ h/Jahr} 
\end{eqnarray}
Die \eur{40} wird der Pauschalbetrag für alle Projektmitarbeiter sein.
\\Die Amortisationszeit setzt sich aus der Division der Gesamtkosten des Projektes und der jährlichen 
Einsparung zusammen.
\\\\$\frac{\eur{1.264,68}}{\eur{1.411,85}\mbox{/Jahr}} \approx 0,90 \mbox{ Jahre} \approx 47 \mbox{ Wochen}$.\\
\\Dadurch ergibt sich eine jährliche Einsparung von 
\begin{eqnarray}
25,67 \mbox{ h} \cdot \eur{(40 + 15)}{\mbox{/h}} = \eur{1.411,85}
\end{eqnarray}

\subsection{Nicht-monetärer Nutzen}
\label{sec:Nicht-monetärer Nutzen}
Für das Projekt werden die Produkte Authelia, Authentik, Microsoft Azure AD und Sitecar, zur Implementierung in Erwägung gezogen. 
Wobei mittels einer Nutzwertanalyse, welche im Kapitel \ref{sec:Authentifizierungs-Tool}: \nameref{sec:Authentifizierungs-Tool} zu sehen ist, der 
Sachverhalt durch eine Entscheidungsmatrix dargestellt wird.
\\Da die Ergebnisse der Wirtschaftlichkeitsanalyse bereits eine ausreichende Begründung für die Umsetzung des Projekts bieten, 
ist es an dieser Stelle nicht notwendig, eine eingehende Untersuchung der nicht-monetären Vorteile vorzunehmen.
\\Ohne der Einführung eines Authentifizierungs-Tools wird die Sicherheit der angebotenen Dienste nicht geboten und das Risiko des 
Datenverlustes gewährleistet. Um das Risiko zu minimieren, soll durch die Nutzwertanalyse ein Ergebnis und die Entscheidungsfindung 
der jeweiligen Authentifizierungsmethode erleichtert werden.

\subsection{Abweichungen vom Projektantrag}
\label{sec:AbweichungenProjektantrag}
Die im Projektantrag mit in \textit{Auflage genehmigt}en Inhalte, erfordern Änderungen in der 
Projektdurchführung und -dokumentation.
\\Erforderliche Änderungen:
\begin{itemize} [label=--]
	\item gesamte Zeitplanung - von 35 auf \textit{40} Stunden gestreckt
	\\Folgen der Zeitänderungen:
	\begin{itemize}
		\item Stundenplanung - Planungsphase von 5 auf \textit{8} Stunden
		\item Stundenplanung - Implementierungsphase von 14 auf \textit{16} Stunden
		\item Stundenplanung - Tesphase von 6 auf \textit{4} Stunden
		\item Stundenplanung - Dokumentation von 8 auf \textit{6} Stunden 
	\end{itemize}
	\item Zeitplanung/ Planungsphase - \textit{Klärung der Projektziele} nach \textit{vorn} der Phase geschoben
	\item Zeitplanung/ Dokumentation - \textit{Benutzerdokumentation} statt Entwicklerdokumentation
\end{itemize}

Nicht erforderliche Änderungen:
\begin{itemize} [label=--]
	\item Implementierungsphase/ Installation und Konfiguration von Sitecars - \textit{Authentik} statt Sitecars
	\\Eine genaue Schilderung, weshalb sich für Authentik anstatt Sitecars entschieden wurde, befindet sich im 
	Kapitel \ref*{sec:Authentifizierungs-Tool} \nameref{sec:Authentifizierungs-Tool}
\end{itemize}

\subsection{Entwicklungsprozess}
\label{sec:Entwicklungsprozess}
Das Projekt unterteilt sich in einem überschaubaren, zeitlich und inhaltlich begrenzten Entwicklungsprozess 
mit gesondert eingeteilten Phasen, die nach- und voneinander aufbauen. So wird eine Sicherstellung der Schritt- 
für Schritt-Fertigstellung der jeweiligen Phasen und Übersicht garantiert. Die klaren und umfassend 
definierten Anforderungen des Projekts ermöglichen eine strukturierte Herangehensweise, die auf 
vorhersehbaren Bedingungen basiert und einen geradlinigen Fortschritt gewährleistet. Auch ist der 
Umfang des Projektes zeitlich eingegrenzt und besitzt eine strukturierte Vorgehensweise mit einer 
geringen Interaktion zum und mit dem Entwicklerteam und des Projektleiters als Kunden.