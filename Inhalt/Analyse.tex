% !TEX root = ../Projektdokumentation.tex
\section{Analyse der Reden}
\label{sec: Analyse der Reden}

\subsection{Redeanalyse}
\label{sec: Redeanalyse}
Im folgenden Kapitel werden zwei Reden von bekannten Politikern analysiert. \\Es handelt sich zum einen um die Antrittsrede von Donald Trump zu seiner Präsidentschaft aus dem Jahr 2017 und zum anderen um die Rede unserer ehemaligen Bundeskanzlerin Angela Merkel zur Coronakrise aus dem Jahr 2020.
\\Interessant wird es in diesem Punkt, ob und inwieweit zwei unterschiedliche Persönlichkeiten, die sich individuell an den eigenen sprachwissenschaftlichen Eigenschaften unterscheiden und rhetorische Mittel in Ihren Reden gegenüber der Zuhörerschaft einsetzen und was als primäres Ziel verwendet wird. \\Die Auswahl der beiden Reden erfolgte bedingt der erneut stattfindenden Wahl diesen Jahres in den Vereinigten Staaten 2024 mit dem fast feststehenden Spitzenkandidaten der Republikaner Donald Trump und der noch immer andauernden Bedrohung bzw. Einflussnahme auf das tägliche Leben durch das Coronavirus und deren Umgang damit.


\subsection{Antrittsrede von Donald Trump} 
\label{sec: Antrittsrede von Donald Trump}
Donald Trumps Antrittsrede nach der Präsidentschaftswahl am 20.01.2017 fand vor der Westfront des Kapitols in Washington, D.C. statt und 
ist ein Paradebeispiel einer Rede, die nicht nur überzeugend ist, 
sondern auch von stilistischen Mitteln geprägt. \\Durch Trumps einfache, gezielte und fassbare Sprache 
befindet sich jeder Bürger in der Lage diese inhaltlich zu verstehen, als auch emotional mitverfolgen 
zu können. So nutzt er, gleich nach seiner Danksagung, die Phraseologie \textit{„wir“} oder \textit{„uns“}, um das 
Gemeinschaftsgefühl zu betonen. Dabei spricht er gleichzeitig die komplette amerikanische Bevölkerung 
an und greift auf die Empathie dieser mittels des Gebrauchs von Anaphern zurück. Diese Art der 
Verwendung der Betonung von Einheit und Gemeinschaft nutzt Trump, um im Zuge der amerikanischen 
Identität, die Bedeutung der Einbringung des Volkes zu verstärken. Kurze Zeit später sagt er 
\textit{„Wir werden Herausforderungen begegnen. Wir werden uns Härten stellen müssen. Aber wir werden die 
Aufgabe erledigen.“}. Er verfolgt ein klares, vordefiniertes Ziel und möchte den gesetzten Plan 
bekräftigen, was seine Entschlossenheit unterstreicht. Parallelismen, wie die Wortgruppe 
\textit{„Wir werden …“} sind bei Trump keine Seltenheit und treten häufiger auf, um das Ziel seiner Rede 
jedem Bürger der vereinigten Staaten mitteilen zu können, als auch die Aussagekraft und Klarheit 
seiner Versprechen für die Zukunft Amerikas zu bestätigen. Er geht auf die Schere von Arm und Reich 
ein, \textit{„zu lange hat eine kleine Gruppe die Vorteile der Regierung genossen, während das Volk die 
Kosten zu tragen hatte“} und verspricht dadurch dem Volk dieses Prinzip der Verwendung von Hyperbeln, 
den aktuellen politischen und gesellschaftlichen Zustand ändern zu wollen. Dergleichen übertreibt 
Donald Trump mit der Größe der privilegierten Minderheit, um die Dominanz der Regierung zu 
bekräftigen, mit dem Ziel die Wiederherstellung von Arbeitsplätzen zu gewährleisten und die 
entstandene Kluft schließen. Bedingt diesen Handelns, bestätigt er den Einsatz durch die 
parataktischen, kurzen, und prägnanten Wortgruppen \textit{„Dies ist Ihr Tag“, „Dies ist Ihre Feier“, 
„…. und dies ist Ihr Land“}. Auffallend sind die Wortwiederholungen und Einbringungen der direkten 
Ansprache und Integration der Bevölkerung. Der Zugriff auf das regelmäßige Einbeziehen des 
Gemeinschaftsgefühls, fesselt die Aufmerksamkeit und verdeutlicht das Verständnis und die bereits 
erwähnte Empathie der Bevölkerungen gegenüber Trump als Präsident der vereinigten Staaten selbst. 
Er möchte das \textit{„amerikanische Gemetzel“} stoppen und weist auf die vorhandene Kriminalität, Gewalt und 
die Probleme der USA hin. Mit dieser Metapher und gleichzeitig Hyperbel betont er die 
Entschlossenheit, etwas verändern zu wollen und seine Zuhörerschaft zu animieren ihm mittels des 
Aufgreifens der Dramatisierung zuzustimmen. Dieser dysphemistischer Ausdruck verdeutlicht die 
Schwere der Situation und kritisiert indirekt die vorherige Regierung, geleitet durch Barack Obama. 
Trump möchte ein besseres Amerika als je zuvor, indem er verdeutlicht, das Personalpronomen \textit{„wir“} 
immer wieder einsetzt und dieses auf das Amerika, sich selbst und die Bevölkerung bezieht – \textit{„Wir sind eine Nation, und ihre Schmerzen sind unsere Schmerzen. Ihre Träume sind unsere Träume, und ihr Erfolg wird unser Erfolg sein.“}. Diese symmetrische Anordnung des Satzbaus spiegelt die Identität der Nation wieder, was das Mitgefühl und die Zugehörigkeit ausdrücken und sich jeder amerikanischer 
Bürger angesprochen fühlen soll. Im letzten Teil seiner Antrittsrede nutzt er die stilistischen 
Mittel von Parataxen, Wortwiederholungen und Hyperbeln in Kombination der biblischen Referenz und 
betont bezugnehmend darauf, die göttliche Unterstützung seines Vorhabens – \textit{„Die Bibel zeigt uns, 
wie gut und wohltuend es ist, wenn Gottes Volk in Einigkeit zusammenlebt.“}. Durch diese Suggestion 
vermittelt Trump Zusammenarbeit und appelliert an den göttlichen Glauben seines Publikums und schafft 
einen moralischen und biblischen Rahmen zu religiösen Überzeugungen. Mit \textit{„Gott segne Amerika“} beendet 
er seine Rede und verdeutlicht durch diesen Abschluss, dass Gottes Segen über Amerika Hoffnung und 
Unterstützung vermitteln wird. \\Zusammenfassend ist zu sagen, dass Donald Trump durch seine 
vereinfachte und wiederholende Wortwahl, die Empathie und Zustimmung der amerikanischen Bevölkerung 
für sich gewinnen konnte.
% \cleardoublepage

\subsection{Coronarede von Angela Merkel} 
\label{sec: Coronarede von Angela Merkel}
Bei dieser Rede handelt es sich um die Fernsehansprache von Angela Merkel in ihrer Funktion als Bundeskanzlerin am 18.03.2020 zur damaligen aktuellen Coronalage. Die Rede wurde im Fernsehen übertragen und richtete sich an die Bundesbürger Deutschlands.
\\Sie beginnt in ihrer Rede mit einer Beschreibung der zu diesem Zeitpunkt aktuellen Lage und berichtet über die bis dahin geschehenen Ereignisse, um die Aufmerksamkeit der Zuschauer zu erhalten. Sie konkretisiert zwar dann mit negativen Beispielen und erzeugt mit dem Indefinitpronomen \textit{„jeder“} auch für diese Situation ein Gemeinschaftsgefühl.
Wichtig, in dieser Thematik für Angela Merkel, ist die in der sie sich gerade befindenden Situation als die wegweisenden Bundeskanzlerin, anerkannt und von möglich allen Bürgerinnern erhört zu werden.
Mit dem Erörtern des Wertesystems einer Demokratie, welche von Transparenz geprägt ist, beginnt sie ein Vertrauensverhältnis ihrer Zuhörerschaft aufzubauen und stellt sich zugleich als Kümmerin dar. Um klarzustellen, dass dieser Weg nicht allein zu gehen ist, holt sie über einen positiven Apell mit einer Wiederholung \textit{„…Ihre Aufgabe, … Es ist ernst. Nehmen Sie es auch ernst.“} ihre Zuhörerschaft mit ins Boot. Um die Dringlichkeit zu unterstreichen, nutzt sie im Folgenden das Stilmittel des Correctivo. Zwar ist in einer vorbereiteten Rede die unmittelbare Korrektur einer gerade getroffenen Aussage eher ungewöhnlich, so wird aber zu dem geschichtlichen Bezug zum Zweiten Weltkrieg über die Wiedervereinigung hinaus, die Bedeutung besonders hervorgehoben.
\\Im Anschluss informiert Angela Merkel wieder in einer sachlichen informierenden Sprache, um wieder die tatsächliche Tragweite über das Erzeugen eines Bildes mit persönlicher Nähe klarzustellen: \textit{"Patienten in Krankenhäusern sind keine Statistiken sondern Menschen"}. Sie schafft es über die die hervorgebrachte Dankbarkeit \textit{„von ganzem Herzen“} an alle Tätigen im Gesundheitswesen und später aufgeführten Mitarbeiter im Lebensmittelhandel Gefühl zu zeigen. Durch die Verwendung von Pathos werden die Zuhörer in ihren Emotionen gepackt und mit in den Bann gezogen. Dies ist dahingehend besonders, da gerade Merkel in ihren typischen Ansprachen bzw. Reden jenes rhetorische Mittel vermissen ließ. Was in diesem Zusammenhang sehr wichtig ist, um die Bevölkerung auf die anstehenden Einschränkungen einzuschwören, ist Glaubwürdigkeit. Dies gelingt ihr durch den persönlichen Bezug auf ihre Biographie. 
\\Wichtig an dieser Stelle ist, dass trotz der angespannten Situation positive Signale gesetzt werden. Würde der Redner keine entsprechenden Punkte finden, führt die negative Stimmung zur Resignation. Hier kommt Angela Merkels bekannter Führungsstil zugute. \textit{„Ich versichere Ihnen“} In ihrer sachlichen Art fängt sie die Gemeinschaft wieder ein und spricht mit \textit{„uns“} auf der Gleichstellungsebene. Was in diesem Zusammenhang bemerkenswert ist, dass sie im gleichen Kontext nicht von Krise, sondern von Prüfung spricht. Zwar geht es um eine \textit{„schwere Prüfung“}, so kann man eine solche, wenn man es möchte, bestehen und strahlt somit Zuversicht aus. Weiter versteht sie negative Beispiele, wie leere Lebensmittelregale positiv zu verpacken, verbunden mit einem weiteren Appell an die Vernunft. Damit sie ihre gewonnene Gemeinschaft nicht verliert, macht sie staatliche Maßnahmen von allen abhängig, \textit{"so muss jetzt auch jede und jeder helfen"}. Durch eine klare Wortwahl und weitere Wiederholungen (ernst nehmen), dreht Angela Merkel die negative Stimmung mit der Herausforderung an alle zurück in das Positive. Durch die Verwendung das mit negativen Erlebnissen verbundene Wort \textit{„verbundbar“}, ergibt sich eine Gemeinsamkeit, die zur Stärkung führt. 
\\Mit einem weiteren stilistischen Mittel, dem Paradoxon, erhöht sie das Mitdenken und die bessere Verinnerlichung: \textit{"aus Rücksicht zueinander (zusammenrücken) müssen wir Abstand halten"}. Mit der Einbringung von Empathie, rückt sie die notwendigen Einschränkungen näher, welche mit einer Vielzahl von Beispielen untermauert werden und zeigt zugleich Lösungen auf. Durch eine stimmige Argumentation in Form von Verweisen auf die derzeit bekannten wissenschaftlichen Fakten, verbunden mit einer deutlichen Erklärung des Ziels der Politik und Forschung, bringt Sie einen logischen Bestandteil in ihre Rede, durch welchen aufgezeigt werden soll, wie die Ausbreitung des Virus zu verlangsamen ist.
\\Zum Ende der Rede beschwört sie mithilfe der Wiederholung der wichtigsten Schlagwörter den Gemeinsinn und Zusammenhalt. Ihre Botschaft an das Volk schließt mit dem sehr emotionalen Schlusssatz: \textit{„Passen sie gut auf sich und auf ihre Liebsten auf. Ich danke Ihen.“}