\section{Test- und Abnahmephase}
\label{sec:Testphase}

\subsection{Überwachung der Laufzeit der Dienste}
\label{sec:Überwachung der Laufzeit der Dienste}
Die Überwachung der Laufzeit der eingetragenen Dienste passiert über das Dashboard von Authentik.
\\Um jeden einzelnen Dienst schlussendlich zu überprüfen, erfolgte über den NGinx Reverse Proxy Manager ein jeweiliger Zugriff auf alle Dienste, um 
zu testen, ob sich Authentik auch wirklich vor den Dienst schaltet. Zusätzlich wird das Monitoring-Tool Uptime Kuma verwendet, um die 
Laufzeit aller Dienste zu überprüfen.

\subsection{Überprüfung/ Beseitung von Fehlern}
\label{sec:Überprüfung/ Beseitung von Fehlern}
Nach der erfolgreichen Durchführung des Projektes wurden die \nameref{sec:Funktional}en und \nameref{sec:Nicht-Funktional}en Anforderungen aus der 
\nameref{sec:Anforderungsanalyse} getestet. Das Beseitigen von Fehlern in der Entwicklungsumgebung erfolgte während der Durchführungsphase 
mittels des Plan-Do-Check-Act-Zyklus, in welchem die Fehler, \zB Tippfehler gleich während der Implementierung behoben wurden.
Aufgrund dieser Vorgehensweise ist ein kontinuierlicher Verbesserungsprozess garantiert.


\subsection{Zugriffstests}
\label{sec:Zugriffstests}
Um den Zugriff zu testen, wurden \cite{BoxTests}-Tests durchgeführt, die Authentik auf die Gesamtheit überprüfen und 
sicherstellen, dass die Benutzeranmeldungen und Authentifizierungen erfolgreich verlaufen. Dabei tappen ein Blackbox Test 
Tester gewissermaßen im Dunkeln. Sie wissen nicht, wie eine Software funktioniert, wie sie implementiert ist oder 
aus welchen Komponenten sie besteht. In einem White-Box Test haben die Testpersonen Kenntnisse des Codes und der 
Funktionsweisen einer Software.

\subsection{Abnahme}
\label{sec:Abnahme}

Die Abnahme verlief reibungslos, wodurch das Projekt am 8. November 2023 erfolgreich an Edgar Johann 
Kapler übergeben und in der Produktivumgebung erfolgreich umgesetzt werden konnte.