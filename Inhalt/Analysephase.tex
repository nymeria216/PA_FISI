% !TEX root = ../Projektdokumentation.tex
\section{Analysephase} 
\label{sec:Analysephase}


\subsection{Ist-Analyse} 
\label{sec:IstAnalyse}
Das Projekt, operiert durch die Deloitte Wirtschaftsprüfungsgesellschaft GmbH, verwendet die Cloud-Infrastruktur 
von OVHCloud, um einen Business Hosting Service bereitzustellen. Dabei enthält diese Konfiguration der 
Cloud-Umgebung eine Firewall, welcher eine öffentliche IP-Adresse zugewiesen bekommen hat. Zusätzlich sollte erwähnt werden, 
dass dieses Netzwerk privat ist und andere Services und Instanzen enthält, welche von der Firewall geschützt werden.
Um einen Zugriff auf die Instanzen zu ermöglichen, wird den Command-Line-Interface-Usern die Möglichkeit geboten, 
über die Kontrollinstanz ''control\_node'' die Instanzen hoch- und runterzufahren und die grundlegenden Einstellungen, wie 
zum Beispiel Portzuweisungen, vorzunehmen. Die Graphical-User-Interface-User haben des Weiteren die Chance die Services über 
den Reverse Proxy Manager zu erreichen, indem der Zugriff klassisch mittels einer einfachen Nutzer- und Passworteingabe, 
ohne weiterer Schutzebene erfolgt.



\subsection{Wirtschaftlichkeitsanalyse}
\label{sec:Wirtschaftlichkeitsanalyse}



\subsubsection{\gqq{Make or Buy}-Entscheidung}
\label{sec:MakeOrBuyEntscheidung}
\begin{itemize}
	\item Gibt es vielleicht schon ein fertiges Produkt, dass alle Anforderungen des Projekts abdeckt?
	\item Wenn ja, wieso wird das Projekt trotzdem umgesetzt?
\end{itemize}


\subsubsection{Projektkosten}
\label{sec:Projektkosten}
Die Kosten für die Durchführung des Projekts setzen sich aus den Personal- und Ressourcenkosten zusammen.   

\begin{eqnarray}
	8 \mbox{ h/Tag} \cdot 220 \mbox{ Tage/Jahr} = 1760 \mbox{ h/Jahr}\\
	\eur{1400}\mbox{/Monat} \cdot 12 \mbox{ Monate/Jahr} = \eur{16800} \mbox{/Jahr}\\
	\frac{\eur{16800} \mbox{/Jahr}}{1760 \mbox{ h/Jahr}} \approx \eur{9,55}\mbox{/h}
\end{eqnarray}

Daraus ergibt sich ein Stundenlohn von \eur{9,55}. 
\\Die Durchführungszeit des Projekts beträgt 40 Stunden. 
Dabei sind mögliche Ressourcen der Stromverbrauch, die zu verwendete Hardware und die Räumlichkeiten, sowie das Büromaterial, wie \zB der zu verwendete 
Monitor, die Peripheriegeräte (Maus, Tastatur, etc.) oder das Möbelar, was pauschal mit \eur{15} kalkuliert werden kann. 
Das Brutto-Einkommen eines Auszubildenden im 3. Lehrjahr im Fachbereich Fachinformatik bei der Deloitte Wirtschaftsprüfungsgesellschaft GmbH 
beträgt \eur{1400} pro Monat. 
Für die weiteren Mitarbeiter wird pauschal ein Stundenlohn von \eur{35} angenommen. 
Eine Aufstellung der Kosten befindet sich in Tabelle~\ref{tab:Kostenaufstellung} und sie betragen insgesamt \eur{1394,68}.
\tabelle{Kostenaufstellung}{tab:Kostenaufstellung}{Kostenaufstellung.tex}


\subsubsection{Amortisationsdauer}
\label{sec:Amortisationsdauer}
Die Amortisation beschleunigt sich durch die Verwendung von Docker, GitLab und Authelia, was die Einsparung von Lizenzkosten zur Folge hat. 
Grund dafür ist, dass diese Plattformen eine Open-Source sind und kostenlos genutzt werden können, was zu einer Reduzierung der  
Gesamtbetriebskosten (Total Cost of Ownerships (TCO)) führt. 
Im Vergleich zu einigen kostenpflichtigen Virtualisierungslösungen, wie \zB Microsoft Hyper-V, können also Lizenzkosten eingespart werden. 
Des Weiteren ermöglicht Docker eine Arbeitszeitersparnis durch die einfache Bereitstellung und Verwaltung von Diensten, was die Arbeitszeit für die 
Einrichtung und Wartung von Umgebungen verkürzt.


\subsection{Nicht-monetärer Nutzen}
\label{sec:Nicht-monetärer Nutzen}
Für das Projekt werden die drei Optionen, MFA-SSO, Authelia und Sitecar, zur Implementierung in Erwägung gezogen. Wobei mittels einer 
Nutzwertanalyse, welche im Kapitel~\ref{sec:Architekturdesign}: \nameref{sec:Architekturdesign} zu sehen ist, der Sachverhalt durch 
eine Entscheidungsmatrix dargestellt wird.
\\Da die Ergebnisse der Wirtschaftlichkeitsanalyse bereits eine ausreichende Begründung für die Umsetzung des Projekts bieten, 
ist es an dieser Stelle nicht notwendig, eine eingehende Untersuchung der nicht-monetären Vorteile vorzunehmen.
\\Ohne der Einführung eines Authentifizierungs-Tools wird die Sicherheit der angebotenen Dienste nicht geboten und das Risiko des 
Datenverlustes gewährleistet. Um das Risiko auszulöschen, soll durch die Nutzwertanalyse ein Ergebnis und die Entscheidungsfindung 
der jeweiligen Authentifizierungsmethode erleichtert werden.


\subsection{Anwendungsfälle}
\label{sec:Anwendungsfaelle}
Ein Use Case-Diagramm zur Veranschaulichung des Prozesses der Cloud-Infrastruktur findet sich im \Anhang{app:UseCase}.
In diesem interagiert aus der Sicht eines Projektentwicklers, dieser als Akteur mit dem System, in welchem verschiedene 
Anwendungsfälle existieren. Der Akteur hat direkten Zugriff auf die Firewall und den Nginx Reverse Proxy Manager. Der Zugriff 
über dieFirewall erfolgt verbindlich mit Authelia, während beim NGinx Reverse Proxy Manager der Akteur sich erst dort anmeldet und im 
Anschluss 
hat das Bedürfnis sich am Nginx 
Reverse Proxy Manager einzuloggen und bevor dieser die Möglichkeit erhält, muss er sich mit Authelia authentifizieren. Nach der 
Authentifizierung ist dieser eingeloggt und kann über den NPM auf die verschiedenen Dienste zugreifen, ohne sich erneut via 
Authelia anmelden zu müssen.
