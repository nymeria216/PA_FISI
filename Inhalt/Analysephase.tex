% !TEX root = ../Projektdokumentation.tex
\section{Analysephase} 
\label{sec:Analysephase}


\subsection{Ist-Analyse} 
\label{sec:IstAnalyse}
Das Projekt, operiert durch die Deloitte Wirtschaftsprüfungsgesellschaft GmbH, verwendet die Cloud-Infrastruktur 
von OVHCloud, um einen Business Hosting Service bereitzustellen. Dabei enthält diese Konfiguration der 
Cloud-Umgebung eine Firewall, welcher eine öffentliche IP-Adresse zugewiesen bekommen hat. Zusätzlich sollte erwähnt werden, 
dass dieses Netzwerk privat ist und andere Services und Instanzen enthält, welche von der Firewall geschützt werden.
Um einen Zugriff auf die Instanzen zu ermöglichen, wird den Command-Line-Interface-Usern die Möglichkeit geboten, 
über die Kontrollinstanz ''control\_node'' die Instanzen hoch- und runterzufahren und die grundlegenden Einstellungen, wie 
zum Beispiel Portzuweisungen, vorzunehmen. Die Graphical-User-Interface-User haben des Weiteren die Chance die Services über 
den Nginx Proxy Manager zu erreichen, indem der Zugriff klassisch mittels einer einfachen Nutzer- und Passworteingabe, 
ohne einer weiteren Schutzebene erfolgt.

\subsection{Soll-Analyse}
\label{sec:SollAnalyse}
Das Soll des Projektes ist es die Dienste mittels Authentik durch die MFA zu schützen. Dabei wird vor jedem Service 
der Identiy-Provider Authentik vorgeschalten, sodass nicht nur MFA, sondern auch SSO passiert und die User automatisch 
authentifiziert werden und das Schutzziel Authentizität erfüllt wird.
\\Schlussfolgernd ist zu sagen, dass der Benutzer sich vor jeder Anmeldung bei einem Service sich erst bei Authentik 
anmeldet, um auf den gewünschten Dienst zugreifen zu können.



\subsection{Wirtschaftlichkeitsanalyse}
\label{sec:Wirtschaftlichkeitsanalyse}
Durch die schon vorhandene Cloud-Infrastruktur entstehen keine weiteren Kosten. Bedingt dessen, dass die 
OVH-Cloud-Infrastruktur zu einem Fix-Preis pro Instanz gemietet wird und keine weiteren Instanzen für die Implementierung 
des Tochter-Projektes erforderlich sind, bleiben die Kosten unverändert. An Ressourcen wird Speicher und mehr 
Rechenleistung benötigt, welche auf der vorhandenen Instanz zur Verfügung stehen und die beiden Faktoren das Arbeiten 
der Instanz nicht beeinträchtigen und resultierend daraus keine größere Instanz notwendig ist. Die wirklich zu entstehenden 
Kosten sind ausschließlich Personal- und Materialkosten, wobei letztere aus der Nutzung des Büromöbelars zusammengefasst wird. 
\\Durch die Einführung von MFA durch Authentik in der Cloud-Infrastruktur werden besonders die Schutzziele der 
Einhaltung der Integrität und Vertraulichkeit der Daten auf den jeweiligen Diensten eingehalten. So wird die Sicherheit 
erheblich verbessert und trägt dazu bei, unbefugten Zugriff, Datenverluste und Betrug zu verhindern. Dieser Schutz 
vor Sicherheitsverletzungen kann erhebliche finanzielle Auswirkungen haben, da die Wiederherstellungskosten vermieden 
werden können. Des Weiteren erfolgen Kostenersparnisse in dem Punkt des Verhinderns der Passwort-Resets durch das Anfragen des 
administrativen Supports, bei welchem Zeit und daraus Kosten entstehen, die vermieden werden können. Durch die Implementierung 
kann den Nutzern ein sicherer und bequemerer Zugriff gewährleistet und die Kosten gesenkt werden.


\subsubsection{\gqq{Make or Buy}-Entscheidung}
\label{sec:MakeOrBuyEntscheidung}
In der Entscheidungsmatrix für die Zielplattform aus dem Kapitel~\ref{sec:Authentifizierungs-Tool}: \nameref{sec:Authentifizierungs-Tool}
auf der Seite \pageref{sec:Authentifizierungs-Tool} sind einige alternative Produkte sichtbar, welche wie Authentik, 
implementiert werden müssen.


\subsubsection{Projektkosten}
\label{sec:Projektkosten}
Die Kosten für die Durchführung des Projekts setzen sich aus den Personal- und Ressourcenkosten zusammen.   

\begin{eqnarray}
	8 \mbox{ h/Tag} \cdot 220 \mbox{ Tage/Jahr} = 1760 \mbox{ h/Jahr}\\
	\eur{1400}\mbox{/Monat} \cdot 12 \mbox{ Monate/Jahr} = \eur{16800} \mbox{/Jahr}\\
	\frac{\eur{16800} \mbox{/Jahr}}{1760 \mbox{ h/Jahr}} \approx \eur{9,55}\mbox{/h}
\end{eqnarray}

Anhand der oben genannten Formel ergibt sich ein Stundenlohn von \eur{9,55}. 
\\Die Durchführungszeit des Projekts beträgt 40 Stunden. 
Dabei sind mögliche Ressourcen, wie der Stromverbrauch, die zu verwendende Hardware und die Räumlichkeiten, sowie das Büromaterial, 
wie \zB der zu nutzende Monitor, die Peripheriegeräte (Maus, Tastatur, etc.) oder das Möbelar, was pauschal mit \eur{15,00} kalkuliert werden kann. 
Das Brutto-Einkommen eines Auszubildenden im 3. Lehrjahr im Fachbereich Fachinformatik bei der Deloitte Wirtschaftsprüfungsgesellschaft GmbH 
beträgt \eur{1400,00} pro Monat. 
Für die weiteren Mitarbeiter werden pauschale Beträge zur Berechnung des Stundensatzes genutzt, aus den Gründen, dass die 
Deloitte Wirtschaftsprüfungsgesellschaft GmbH die Kosten pro Stunde nicht preisgeben möchte. Duale Studenten werden pauschal mit 
\eur{15,00}, während die Manager mit \eur{50,00} pro Stunde berechnet werden. Bei jeweils beiden addiert sich die Summe der 
Ressourcenkosten auf. Die Gesamtkosten, dargestellt in der Tabelle~\ref{tab:Kostenaufstellung} betragen \eur{1228,78}.
\tabelle{Kostenaufstellung}{tab:Kostenaufstellung}{Kostenaufstellung.tex}


\subsubsection{Amortisationsdauer}
\label{sec:Amortisationsdauer}
Die Amortisation beschleunigt sich durch die Verwendung von Docker, GitLab und Authentik, was die Einsparung von Lizenzkosten zur Folge hat. 
Grund dafür ist, dass diese Plattformen Open-Source sind und kostenlos genutzt werden können, was zu einer Reduzierung der  
Gesamtbetriebskosten (Total Cost of Ownerships (TCO)) führt. 
Im Vergleich zu einigen kostenpflichtigen Virtualisierungslösungen, wie \zB Microsoft Hyper-V, können also Lizenzkosten eingespart werden. 
Des Weiteren ermöglicht Docker eine Arbeitszeitersparnis durch die einfache Bereitstellung und Verwaltung von Diensten, was die Arbeitszeit für die 
Einrichtung und Wartung von Umgebungen verkürzt.


\subsection{Nicht-monetärer Nutzen}
\label{sec:Nicht-monetärer Nutzen}
Für das Projekt werden die Produkte Authelia, Authentik, Microsoft Azure AD und Sitecar, zur Implementierung in Erwägung gezogen. 
Wobei mittels einer Nutzwertanalyse, welche im Kapitel~\ref{sec:Architekturdesign}: \nameref{sec:Architekturdesign} zu sehen ist, der 
Sachverhalt durch eine Entscheidungsmatrix dargestellt wird.
\\Da die Ergebnisse der Wirtschaftlichkeitsanalyse bereits eine ausreichende Begründung für die Umsetzung des Projekts bieten, 
ist es an dieser Stelle nicht notwendig, eine eingehende Untersuchung der nicht-monetären Vorteile vorzunehmen.
\\Ohne der Einführung eines Authentifizierungs-Tools wird die Sicherheit der angebotenen Dienste nicht geboten und das Risiko des 
Datenverlustes gewährleistet. Um das Risiko zu minimieren, soll durch die Nutzwertanalyse ein Ergebnis und die Entscheidungsfindung 
der jeweiligen Authentifizierungsmethode erleichtert werden.


\subsection{Anwendungsfälle}
\label{sec:Anwendungsfaelle}
Ein Use Case-Diagramm zur Veranschaulichung des Prozesses der Cloud-Infrastruktur findet sich im \Anhang{app:UseCase}.
In diesem interagiert der Akteur aus der Sicht eines Projektentwicklers mit dem System, in welchem verschiedene 
Anwendungsfälle existieren. Der Akteur meldet sich über den Authentik Server bei der Firewall und dem Nginx Proxy Manager an. 
Nach der erfolgreicher Anmeldung mit der MFA hat dieser die Möglichkeit auf die dann zur Verfügung stehenden Dienste vom 
Nginx Reverse Proxy Manager aus zuzugreifen.
\\Hat sich der Akteur mit dem Authentik Server verbunden, der aus dem dem eigentlichen Server (Authentik Server Core) und dem 
integrierten Außenposten (Embedded oupost) besteht. Einkommende Anfragen an den Server-Containern werden an den Authentik Server Core 
or dem Embedded oupost geroutet. Der Authentik Core Server verarbeitet den großteil der Logik von Authentik, wie \zB API- und/oder 
SSO-Anfragen, während der Embedded outpost die Verwendung von Proxy-Anbietern ermöglicht, ohne dass eine separate Außenstelle 
eingerichtet werden muss. Der Hintergrundarbeiter (Background Worker) führt Hintergrundaufgaben aus, wie das Senden von E-Mails, 
oder Benachrichtigen von Ereignisses und alles, was im Frontend sichtbar ist. Authentik nutzt PostgreSQL, um alle seiner 
Konfigurationen und Daten zu speichern. Redis wird als Message-Queue und Cache verwendet.