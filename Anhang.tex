% !TEX root = Projektdokumentation.tex
\appendix
\section{Anhang}
\label{sec:Anhang}

\subsection*{Abkürzungsverzeichnis}
\label{sec:Abkürzungsverzeichnis}
\addcontentsline{toc}{subsection}{Abkürzungsverzeichnis}
\begin{acronym}[WWWWW]
	\acro{TAL}{Tech Accelerator Leipzig}
	\acro{SSH}{Secure Shell}
	\acro{B and TCL}{Business \& Technology Ceenter Leipzig}
	\acro{API}{Application Programming Interface}
	\acro{LDAP}{Lightweight Directory Access Protocol}
	\acro{OIDC}{OpenID Connect}
	\acro{CLI}{Command Line Interface}
	\acro{GUI}{Graphical User Interface}
	\acro{ISMS}{Informationssicherheits-Managementsystem}
	\acro{BSI}{Bundesamt für Sicherheit in der Informationstechnik}
	\acro{TOTP}{Time-based One-time Password}
	\acro{SSO}{Single-Sign On}
	\acro{MFA}{Multi Faktor Authentifizierung}
	\acro{IDE}{Integrated Development Environment}
	\acro{TCO}{Total Cost of Ownership}
\end{acronym}
\cleardoublepage

\subsection{Gantt-Diagramm und Meilensteine}
\label{app:Gantt}
\includegraphicsKeepAspectRatio{Anhang/Gantt.pdf}{1}
\clearpage

\subsection{Detaillierte Zeitplanung}
\tabelleAnhang{ZeitplanungKomplett}
\clearpage

\subsection{Use Case-Diagramm}
\label{app:UseCase}
\begin{figure}[htb]
\centering
\includegraphicsKeepAspectRatio{Anhang/UseCaseCloudInfra.png}{1}
\caption{Use Case-Diagramm}
\end{figure}
\clearpage

\subsection{Sequenzdiagramm CLI Zugriff auf die Instanzen}
\label{app:Sequenzdiagramm CLI Zugriff auf die Instanzen}
\includegraphicsKeepAspectRatio{Anhang/SequenzDiagrammCLI.png}{1}
\clearpage

\subsection{Cloud-Infrastruktur}
\label{app:Cloud-Infrastruktur}
\begin{figure}[htb]
    \centering
    \includegraphicsKeepAspectRatio{Anhang/CloudInfra.png}{0.8}
\end{figure}
\pagebreak

\subsection{docker-compose.yml}
\label{app:docker-compose.yml}
\includegraphicsKeepAspectRatio{Anhang/docker-compose.pdf}{0.9}

\subsection{.env}
\label{app:dotenv}
\includegraphicsKeepAspectRatio{Anhang/(.)env.pdf}{0.9}

\subsection{Docker-Befehle}
\label{app:dockercommands}
\begin{figure}[ht]
    \centering
    \includegraphics[scale=0.4]{Bilder/Authentik-Doc/DP_00_DockerPull.png}
    \caption{Docker Pull}
  \end{figure}
  
  \vspace{0.5cm} % Adjust vertical space as needed
  
  \begin{figure}[ht]
    \centering
    \includegraphics[scale=0.4]{Bilder/Authentik-Doc/DP_01_DockerComposeUp.png}
    \caption{Docker Compose Up}
  \end{figure}

\subsection{Proxy Host Konfiguration}
\label{app:ProxyHostConfig}
\includegraphicsKeepAspectRatio{Anhang/ProxyHostKonfiguration.pdf}{0.9}

\subsection{Authentik-Konfiguration}
\label{app:AuthentikConfig}
% \includepdf[pages={1-3}, scale=0.9]{Anhang/Authentik-Konfiguration.pdf}
\includegraphicsKeepAspectRatio{Anhang/AuthentikKonfig1.png}{0.9}
\clearpage
\includegraphicsKeepAspectRatio{Anhang/AuthentikKonfig2.png}{0.9}
\clearpage
\includegraphicsKeepAspectRatio{Anhang/AuthentikKonfig3.png}{0.9}
\clearpage

\subsection{NGinx Konfiguration}
\label{app:CustomNGinxConfig}
\includegraphicsKeepAspectRatio{Anhang/Custom_NGinx_Configuration.pdf}{0.9}
\clearpage
\begin{figure}[htb]
    \centering
    \includegraphicsKeepAspectRatio{Bilder/Authentik-Doc/NS_03_ChangeIPInCustomNGinxConfig.png}{0.4}
    \captionbelow{Proxy Host Konfiguration - Einfügen des Code-Snippets in den Bereich \textbf{Advanced} 
    und erfolgt das Ändern der IP-Adresse mit dem zugehörigen Port}
\end{figure}

\subsection{TOTP-Einrichtung}
\label{sec:TOTPConfig}
\includegraphicsKeepAspectRatio{Anhang/TOTP-Einrichtung.pdf}{0.9}

\subsection{Benutzerdokumentation}
\label{app:Benutzerdokumentation}
\includegraphicsKeepAspectRatio{Anhang/Benutzerdokumentation.pdf}{0.9}

\subsection{Testprotokoll}
\label{app:Testprotokoll}
\includegraphicsKeepAspectRatio{Anhang/Testprotokoll.pdf}{0.9}

\subsection{Übernahmeprotokoll}
\label{app:Übernahmeprotokoll}
\includegraphicsKeepAspectRatio{Anhang/Übernahmeprotokoll.pdf}{0.9}

% Abbildungsverzeichnis ------------------------------------------------------
\phantomsection
\listoffigures
\cleardoublepage

% Tabellenverzeichnis
\phantomsection
\listoftables
\cleardoublepage

% Literatur ------------------------------------------------------------------
\clearpage
\renewcommand{\refname}{Literaturverzeichnis}
\bibliography{Bibliographie}
\bibliographystyle{Allgemein/natdin} % DIN-Stil des Literaturverzeichnisses
% !TEX root = Projektdokumentation.tex
\clearpage
\addsec{Selbstständigkeitserklärung}

% Hinweis: die eidesstattliche Erklärung ist ggfs. an die Richtlinie der IHK anzupassen
Selbstständigkeitserklärung:
\\Hiermit verischere ich, \autorName, dass die vorliegende Arbeit 
% Thema
\begin{quote}
\textit{\kompletterTitel}
\end{quote}
selbständig und nur mit Hilfe der angegebenen Quellen verfasst habe.

\abgabeOrt, den \abgabeTermin


\rule[-0.2cm]{5.5cm}{0.5pt}

\textsc{\autorName}
% \includegraphicsKeepAspectRatio{Anhang/Selbstständigkeitserklärung.pdf}{0.9}
\clearpage
\cleardoublepage