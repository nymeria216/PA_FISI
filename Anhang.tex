% !TEX root = Projektdokumentation.tex
\appendix
% \section{Anhang}
% \label{sec:Anhang}

\includepdf[pages=1,scale=0.7,pagecommand={\section{Anhang}\subsection{Antrittsrede von Donald Trump}\label{ch:turing_materials}\hfill}]{Anhang/Trump_Antrittsrede_2017.pdf}
\includepdf[pages=2-,scale=.7,pagecommand={},linktodoc=true]{Anhang/Trump_Antrittsrede_2017.pdf}

\includepdf[pages=1,scale=0.7,pagecommand={\subsection{Coronarede von Angela Merkel}\label{ch:turing_materials}\hfill}]{Anhang/Merkel_Coronarede.pdf}
\includepdf[pages=2-,scale=.7,pagecommand={},linktodoc=true]{Anhang/Merkel_Coronarede.pdf}
\subsection{Begriffserklärung}
\label{sec:Begriffserklärung}
% \addcontentsline{toc}{subsection}{Begriffserklärung}
\begin{description}
  \item[linguistische Eigenschaften] - sprachliche Merkmale
  \item[Phraseologie] - Redensart einer Sprache
  \item[Paralellismen] - Wortwiederholungen
  \item[Hyperbel] - Übertreibungen
  \item[Metapher] - bildliche Sprache
  \item[dysphemistischer Ausdruck] - negativer Ausdruck/ Äußerungen
  \item[Parataxen] - unabhängige kurze Sätze
  \item[Correctivo] - Stilmittel, wenn der Redner sich selbst berichtigt
  \item[Paradoxon] - Stilmittel, mit einer wiederprüchlichen Behauptung 
  \item[Anapher] - Stilmittel, Wiederholung eines Wortes am Satzanfang 
  \item[populistische Züge] - Stimmung hervorrufen  
\end{description}
\cleardoublepage

\renewcommand{\refname}{Literaturverzeichnis}
\bibliography{Bibliographie}
\bibliographystyle{Allgemein/natdin} % DIN-Stil des Literaturverzeichnisses

\textbf{Buch-Quellen:}
\begin{itemize}
    \item Name: Rhetorik zur Eiführung, Autor: Melanie Möller (Kapitel: 2.1 und 2.4 )
    \item Name: Reden können in der Demokratie/1. Grundlagen rhetorischer Kommunikation, Autor: Joachim Detjen (Kapitel: 2.4 und 2.6)
\end{itemize}

\textbf{Internet-Quellen:}\\
\\Die 3 Säulen der Rhetorik für bewegende Landing-Page-Optimierung
\begin{itemize}
    \item Kapitel: 2.3
    \item Zugriffszeit: 28.12.2023, 14:47
    \item Link: https://blog.hubspot.de/website/landing-page-optimierung-nach-aristoteles
\end{itemize}
Ethos in Beispielsätzen
\begin{itemize}
    \item Kapitel: 2.3.1
    \item Zugriffszeit: 01.01.2024, 15:12
    \item Link: https://www.collinsdictionary.com/de/satze/deutsch/ethos
\end{itemize}
Ethos. Pathos. Logos
\begin{itemize}
    \item Kapitel: 2.3.1 und 2.3.2 
    \item Zugriffszeit: 01.01.2024, 15:34  
    \item Link: https://starkmitworten.de/ethos-pathos-logos/
\end{itemize}
Logos,Ethos und Pathos: Die drei Säulen erfolgreicher Rhetorik 
\begin{itemize}
    \item Kapitel: 2.3.2 und 2.3.3
    \item Zugriffszeit: 01.01.2024, 16:24  
    \item Link: https://bileico.com/blog/logos-ethos-pathos-rhetorik.html
\end{itemize}
Rhetorische Mittel
\begin{itemize}
    \item Kapitel: 2.4.1
    \item Zugriffszeit: 02.01.2024, 18:30
    \item Link: https://www.stagement.com/blog/rede-verbessern-rhetorischestilmittel/
\end{itemize}
Rhetorische Mittel Definition, Liste und Beispiele
\begin{itemize}
    \item Kapitel: 2.4
    \item Zugriffszeit: 02.01.2024, 18:54
    \item Link: https://www.bachelorprint.de/wissenschaftliches-schreiben/rhetorische-mittel/ 
\end{itemize}
Rhetorische Mittel: Bedeutung und Beispiele
\begin{itemize}
    \item Kapitel: 2.4
    \item Zugriffszeit: 02.01.2024, 21:52  
    \item Link: https://lessons2go.de/magazin/artikel/rhetorische-mittel-bedeutung-beispiele.html
\end{itemize}
Rhetorische Mittel Liste mit Beispielen und Erklärungen
\begin{itemize}
    \item Kapitel: 2.4
    \item Zugriffszeit: 02.01.2024, 22:17 
    \item Link: https://www.scribbr.de/wissenschaftliches-schreiben/rhetorische-mittel/    
\end{itemize}
Rhetorik im Alltag und Beruf: Warum es sich lohnt, Ihre Fähigkeiten zu verbessern
\begin{itemize}
    \item Kapitel: 2.5
    \item Zugriffszeit: 03.01.2024 14:59  
    \item Link: https://www.openpr.de/news/1244738/Rhetorik-im-Alltag-und-Beruf-Warum-es-sich-lohnt-Ihre-Faehigkeiten-zu-verbessern.html.   
\end{itemize}
Rhetorik: Diese 8 Fehler solltest du vermeiden, um kraftvoll zu kommunizieren
\begin{itemize}
    \item Kapitel: 2.5
    \item Zugriffszeit: 07.01.2024, 17:32  
    \item Link: https://www.stellenanzeigen.de/careeasy/kraftvoll-kommunizieren-sde18286/   
\end{itemize}
Rhetorik Definition
\begin{itemize}
    \item Kapitel: 2.6
    \item Zugriffszeit: 20.02.2024, 16:59  
    \item Link: https://www.rhetorik-homepage.de/
\end{itemize}
Rhetorik für den Alltag
\begin{itemize}
    \item Kapitel: 2.5 und 2.6 
    \item Zugriffszeit: 03.01.2024, 15:27   
    \item Link: https://www.rheacting.de/rhetorik-fuer-den-alltag/  
\end{itemize}
% !TEX root = Projektdokumentation.tex
\clearpage
\addsec{Selbstständigkeitserklärung}

% Hinweis: die eidesstattliche Erklärung ist ggfs. an die Richtlinie der IHK anzupassen
Selbstständigkeitserklärung:
\\Hiermit verischere ich, \autorName, dass die vorliegende Arbeit 
% Thema
\begin{quote}
\textit{\kompletterTitel}
\end{quote}
selbständig und nur mit Hilfe der angegebenen Quellen verfasst habe.

\abgabeOrt, den \abgabeTermin


\rule[-0.2cm]{5.5cm}{0.5pt}

\textsc{\autorName}
% \includegraphicsKeepAspectRatio{Anhang/Selbstständigkeitserklärung.pdf}{0.9}
\clearpage
\cleardoublepage

% \subsection{Antrittsrede von Donald Trump}
% \label{sec: Trumps Rede}
% \includepdf[pages=1-11]{Anhang/Trump_Antrittsrede_2017.pdf}

% \subsection{Rede von Angela Merkel}
% \label{sec: Rede von Merkel}
% \includepdf[pages=1-7]{Anhang/Merkel_Coronarede.pdf}