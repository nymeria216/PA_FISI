% !TEX root = Projektdokumentation.tex

% Es werden nur die Abkürzungen aufgelistet, die mit \ac definiert und auch benutzt wurden. 
%
% \acro{VERSIS}{Versicherungsinformationssystem\acroextra{ (Bestandsführungssystem)}}
% Ergibt in der Liste: VERSIS Versicherungsinformationssystem (Bestandsführungssystem)
% Im Text aber: \ac{VERSIS} -> Versicherungsinformationssystem (VERSIS)

% Hinweis: allgemein bekannte Abkürzungen wie z.B. bzw. u.a. müssen nicht ins Abkürzungsverzeichnis aufgenommen werden
% Hinweis: allgemein bekannte IT-Begriffe wie Datenbank oder Programmiersprache müssen nicht erläutert werden,
%          aber ggfs. Fachbegriffe aus der Domäne des Prüflings (z.B. Versicherung)

% Die Option (in den eckigen Klammern) enthält das längste Label oder
% einen Platzhalter der die Breite der linken Spalte bestimmt.
\begin{acronym}[WWWWW]
	\acro{TAL}{Tech Accelerator Leipzig}
	\acro{B&TCL}{Business & Tech Accelerator Leipzig}
	\acro{API}{Application Programming Interface}
	\acro{VCS}{Version Control System}
	\acro{TOTP}{Time-based One-time Password}
	\acro{SSO}{Single-Sign On}
	\acro{MFA}{Multi Faktor Authentifizierung}
	\acro{IDE}{Integrated Development Environment}
\end{acronym}
